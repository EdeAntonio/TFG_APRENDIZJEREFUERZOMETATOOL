\documentclass[12pt,a4paper]{report}

% Inclusión de paquetes básicos
\usepackage[utf8]{inputenc} % Paquete para lectura de acentos y símbolos.
\usepackage[T1]{fontenc} % Paquete para exportar a PDF
\usepackage[spanish]{babel} % Cambia el idioma al español
\usepackage{graphicx} % Paquete para la inserción de imágenes
\usepackage{geometry} % Permite cambiar los márgenes de la página
\usepackage{setspace} % Permite controlar el interlineado y espacios
\usepackage{titling} % Permite mayor flexibilidad a la hora de hacer la portada.
\usepackage{path} %Para directorios
\usepackage{lscape} % Permite poner páginas en horizontal
\usepackage{pdflscape} % Permite girar la página dentro del PDF
\usepackage{amsmath}
\usepackage{amssymb} % Para incluir matemáticas en el documento.
\usepackage{lmodern} % Fuente Latin Moderna
\usepackage{cite} % Paquete para citas
\usepackage{placeins} % Para evitar figuras en medio de los códigos
\usepackage{hyperref} %Para urls
\usepackage{rotating} %Rotar figura en Anexo

\DeclareMathOperator*{\argmax}{arg\,max} % Declaración de un operador matemático.
% Configuración de márgenes
\geometry{margin=2.5cm}

% Datos del documento
\newcommand{\universidad}{Universidad Politécnica de Madrid}
\newcommand{\facultad}{Escuela Técnica Superior de Ingeniería y Diseño Industrial}
\newcommand{\tituloTFG}{Estudio del aprendizaje por refuerzo en robótica y aplicación práctica en el proyecto Metatool}
\newcommand{\autor}{Enrique de Antonio}
\newcommand{\tutor}{Pfr. Miguel Hernando Gutierrez}
\newcommand{\cotutor}{Dr. Virgilio Gómez Lambo}
\newcommand{\departamento}{Ingeniería Eléctrica, Automática y Física Aplicada}
\newcommand{\lugar}{Madrid}
\newcommand{\fecha}{febrero 2026}
\newcommand{\LogoUPM}{imagenes/LogoUPM.png}
\newcommand{\LogoETSIDI}{imagenes/LogoETSIDI.png}

\newcommand{\codigo}[1]{\texttt{#1}}
\newcommand{\clase}[1]{\texttt{#1}}     % para clases
\newcommand{\metodo}[1]{\texttt{#1}}    % para métodos
\newcommand{\atributo}[1]{\texttt{#1}}  % para atributos
\newcommand{\api}[1]{\texttt{#1}}


\usepackage{listings}
\usepackage{xcolor}

%\begin{lstlisting}[style=mypython, caption={Ejemplo de código Python},  label={lst:codigo-suma}]
\lstdefinestyle{mypython}{ 
    language=Python,
    basicstyle=\ttfamily\small,
    keywordstyle=\color{blue},
    commentstyle=\color{gray},
    stringstyle=\color{teal},
    showstringspaces=false,
    frame=single,
    breaklines=true,
}



% Formato del documento
    % Márgenes
\geometry{a4paper, top=2.5cm, bottom=2.5cm, left=2.5cm, right=2.5cm}

    % Interlineado
\onehalfspacing % Interlineado de 1.5

    % Sangría y espacio entre párrafos
\setlength{\parindent}{1.25cm}
\setlength{\parskip}{0.2cm}

% Inicio del documento
\begin{document}

% Portada
\begin{titlepage}
    \centering

    % Parte superior con logos
    \begin{minipage}{0.45\textwidth}
        \raggedright
        \includegraphics[height=2.5cm]{\LogoUPM}
    \end{minipage}%
    \hfill  
    \begin{minipage}{0.45\textwidth}
        \raggedleft
        \includegraphics[height=2cm]{\LogoETSIDI}
    \end{minipage}

    \vspace{3cm}

    % Parte central con texto
    {\Large \universidad \par}
    \vspace{0.3cm}
    
    {\large \facultad \par}
    \vspace{1.5cm}
    
    {\LARGE \textbf{Trabajo de Fin de Grado} \par}
    \vspace{1.5cm}
    
    {\huge \tituloTFG \par}

    \vspace{3cm}

    \raggedright
    {\large
        \textbf{Autor:} \autor \par
        \textbf{Tutor:} \tutor \par
        \textbf{Cotutor:} \cotutor \par
        \textbf{Departamento:} \departamento \par
    }

    \vfill % empuja lo siguiente al fondo de la página
    \hfill
    {\large \lugar, \fecha \par}
\end{titlepage}
\tableofcontents
\chapter{Introducción}

\section{Contexto y motivación}
Según la página oficial de \emph{Nvidia} \cite{Nvd_def_RL}, una de las mayores impulsoras de esta disciplina, el aprendizaje por refuerzo es una técnica de aprendizaje automático que permite a los robots tomar decisiones basadas en la experiencia. En este trabajo de final de grado se va a estudiar esta doctrina; con el objetivo de entender sus conceptos fundamentales y poder crear (mediante la herramienta de \emph{Nvidia}, \emph{IsaacLab}) distintos entornos capaces de ejecutar este procedimiento.

Las inteligencias artificiales son actualmente una tecnología puntera con una gran cantidad de aplicaciones. Concretamente, el aprendizaje automático se ha aplicado en disciplinas como la medicina, en la generación de reportes de imágenes médicas; como las finanzas, en la reserva de órdenes de compra; o como la energía, en sistemas de refregamiento de bancos de datos \cite{RL_aplications}. En la robótica concretamente, ha tomado un gran protagonismo. En esta disciplina, grandes empresas del sector como \emph{Boston Dynamics} han empezado a utilizar aprendizaje automático en múltiples tareas \cite{BD_RLusage}.

El interés en este proyecto nace de la idea de aplicar esta herramienta dentro del proyecto \emph{ROMERIN}, un robot modular escalador para la inspección de infraestructuras \cite{Romerin_Descrip}.  Debido a la complejidad del aprendizaje por refuerzo, se vio la necesidad de realizar un estudio completo. Esto, añadido a la cesión de recursos del proyecto \emph{MetaTool} en la formación, llevo a colaborar dentro de este proyecto, analizando y revisando código. 

Antes de comenzar el trabajo, en este capítulo, se estudiarán los objetivos y contenidos de este trabajo. De esta forma, se obtendrá una visión clara de las ideas principales y la estructura del documento. En el siguiente apartado, se enumeraran los principales objetivos de este trabajo.

\section{Objetivos del trabajo}
Este proyecto busca realizar un estudio del aprendizaje por refuerzo y la herramienta IsaacLab, para lo cual se fijan dos objetivos principales. En primer lugar, asentar una base teórica fuerte tanto del aprendizaje por refuerzo como la herramienta IsaacLab. Se pretende que futuros estudiantes puedan basarse en ella para realizar trabajos en esta disciplina. En segundo lugar,  realizar labores dentro del proyecto europeo MetaTool; utilizando estas para ganar experiencia.

Para alcanzar estos objetivos, se proponen una serie de objetivos secundarios:
\begin{enumerate}
    \item Obtener una visión general del impacto del aprendizaje por refuerzo en el campo de la robótica.
    \item Estudiar las bases del aprendizaje por refuerzo, centrándose en su estructura, base matemática y sus principales algoritmos.
    \item Explicar el funcionamiento de la herramienta \emph{IsaacLab} para su aplicación en aprendizaje por refuerzo.
    \item Analizar ejemplos de dicha herramienta, para  así profundizar en ella y proveer de una guía práctica para trabajos futuros.
    \item Estudio del proyecto \emph{MetaTool}: Misión y Visión.
    \item Realización de trabajos prácticos en el proyecto con la herramienta \emph{IsaacLab}
    \item Estudio del problema \emph{Sim2Real} y posibles soluciones
    \item Realización de un código para la implementación de políticas.
\end{enumerate}

Para el cumplimiento de estos objetivos, se deberá tener en cuenta todo lo se va abarcar y cómo este alcance se adapta a los objetivos propuestos. En el siguiente apartado, se realizará esto mismo. 

\section{Alcance y limitaciones}
En este TFG cubriremos el proceso para realizar entornos de aprendizaje automático. Al tener este objetivo en mente, en este TFG se podrán encontrar distintos aspectos de esta disciplina. Este trabajo contempla desde los aspectos más fundamentales de la teoría del aprendizaje automático, hasta las distintas estructuras de datos, clases y ficheros que ejecutan y simulan los entornos.

Primero de todo, para situarse dentro del marco del Aprendizaje por Refuerzo en robótica, se revisará el estado actual del arte. Se presentarán los avances más importantes en distintos campos de la robótica; entre ellos la manipulación, la locomoción y otras aplicaciones como drones, navegación o dispositivos de visión.

A continuación, se explicará la teoría fundamental del Aprendizaje por Refuerzo. En una primera instancia, se presentará la estructura principal que se utiliza en esta disciplina y sus partes, entre las que se encuentran los agentes, los entornos y sus interacciones (acciones, observaciones y recompensas). Dentro de este marco teórico se estudiará los procesos de decisión Markov (MDP), en los cuales se asienta la base de los algoritmos que se utilizarán. Una vez estudiado esto, se presentarán algunos de estos algoritmos, desde los más simples (\emph{Monte Carlo}, \emph{TD}) hasta los que se utilizarán en las simulaciones (\emph{PPO}, \emph{SAC}, \emph{A2C}).

Una vez desarrollado el marco teórico, se procederá a introducir la herramienta \emph{IsaacLab}. Después de una primera introducción a la herramienta, se comenzará a explicar sus distintas funcionalidades. Primero, se explicará qué es y cómo se estructuran las simulaciones dentro de la aplicación. Después, se analizarán las dos principales arquitecturas de los entornos: la forma directa y la basada en manejadores. Definidas las arquitecturas, se estudiarán las principales estructuras de datos: las clases y los tensores. Por último, se estudiará como, una vez definidos los entornos, se realiza el aprendizaje y cómo se evalúa el resultado final.

A continuación, se analizará dos casos prácticos de la herramienta \emph{IsaacLab}; cada uno construido a partir de una arquitectura diferente. Para ambos ejemplos, primero, se presentará la tarea escogida y la motivación detrás de esta elección. Seguidamente, se presentará el diagrama de clases que describe el entorno a entrenar. Este diagrama de clases se diseccionará, analizando cada una de las clases, con especial interés en sus atributos y métodos. Desmenuzado el entorno, se estudiará la ejecución del aprendizaje, valorando después el resultado final de esta. Por último, se presentarán algunas mejoras posibles dentro del ejercicio.

Habiendo estudiado las distintas características del RL y la herramienta, se comenzará a intervenir dentro del proyecto MetaTool. En este capítulo, se definirá el contexto del proyecto \emph{MetaTool} y cuál es su principal objetivo. Seguidamente, se concretará las tareas en las cuales se intervendrá y el objetivo de la participación. Para cada caso realizado, se expondrán los problemas afrontados y las soluciones tomadas.

Otro punto importante del aprendizaje por refuerzo que se estudiará será el problema del Sim2Real, que consiste en la aplicación de las políticas entrenadas para el control en el entorno real. Primero, se presentará el concepto de Sim2Real y sus principales desafíos. Seguidamente, se enumerarán y analizarán las distintas técnicas para realizar este trasvase a la realidad. Finalmente, se preparará un código para la implementación de las políticas en robots reales.

En la última parte del trabajo, se expondrán las conclusiones del proyecto en su conjunto, valorando las aportaciones realizadas, las dificultades encontradas y las oportunidades de trabajos futuros.

Este trabajo, debido a la gran extensión de esta disciplina, se dejan de cubrir algunos paradigmas. Por un lado, únicamente se estudia dentro del aprendizaje automático el aprendizaje por refuerzo, dejando fuera el aprendizaje supervisado y no supervisado. Además, pese a que se realice aprendizaje por refuerzo profundo, no se entrará en detalle en la base matemática de sus algoritmos, así como las bibliotecas implementadas con estos. Se estudiarán las características de las redes neuronales, pero no se profundizará en la matemática detrás de ellas.

Este será el alcance completo del trabajo. Sim embargo, se debe tener en cuenta un punto más antes de comenzar con las tareas prácticas: la metodología. En el siguiente apartado se cubrirá este tema.

\section{Metodología}
En este trabajo de final de grado existen principalmente dos líneas: una parte teórica acerca del aprendizaje por refuerzo y una parte práctica mediante programación en Python y finalmente URscript. Por otro lado, la preparación de este documento se ha realizado después de 8 meses realizando tareas de programación e investigación por propia cuenta; o en conjunto con el equipo de investigación del proyecto MetaTool. A continuación, se expondrá la metodología característica de cada apartado.

La introducción, en primer lugar, se ha preparado después de haber realizado la mayoría de las labores teóricas y prácticas del trabajo. Teniendo así una visión general del trabajo global, se han expuesto las distintas características de este y su enfoque general.

Para el estado del arte, al querer mostrar una visión general del estado del aprendizaje por refuerzo en la robótica, se han buscado distintos artículos de investigación sobre esta disciplina y sus aplicaciones prácticas. Al haber realizado este ejercicio después de este periodo de aprendizaje y práctica, se han podido identificar los factores más importantes de cada artículo, así como identificar los artículos más relevantes.

En cuanto al apartado 3, en el cual se exponen los fundamentos teóricos del aprendizaje por refuerzo, se ha seguido la siguiente metodología. En primer lugar, se estudió un curso de aprendizaje por Refuerzo impartido por David Silver \cite{silver_lectures_nodate}. Mediante este curso se obtuvo una visión general de esta disciplina, entendiendo su estructura general y la base para sus algoritmos. Una vez obtenida una visión general, y después de aplicar esta visión en labores prácticas, se estudió más específicamente cada elemento, indagando en distintas fuentes de información. Cabe resaltar dentro de estas fuentes, el libro sobre aprendizaje por refuerzo de Sutton y Barto \cite{sutton_reinforcement_2020}. Sobre este se trabajan la gran mayoría de definiciones formales. 

El 4º apartado, acerca de la herramienta \emph{IsaacLab}, se trató de manera distinta. Al ser \emph{IsaacLab} una herramienta concreta y propiedad de una entidad privada, Nvidia, existe menos diversidad de información acerca de ella. Por tanto, para su estudio, se utilizó principalmente la información contenida en sus fuentes oficiales. En este apartado concreto, se utilizaron principalmente los tutoriales proporcionados por la plataforma para el aprendizaje de su estructura y aplicación, así como los distintos glosarios de las funciones y clases; y su información detallada acerca de la propia plataforma y sus bibliotecas. A esto se le sumo el conocimiento aprendido en el trabajo en conjunto con el equipo MetaTool, en especial con Virgilio Gómez, especialista de la plataforma y líder de la división en la que se trabajó.

El apartado 5 y 6, al constar de un análisis concreto de un ejemplo proporcionado por la plataforma, se han utilizado los conocimientos aprendidos en el apartado anterior. Para realizar este análisis se ha seguido el siguiente proceso. En primer lugar, se analizó todo el ejercicio en su conjunto, estudiando los distintos ficheros que se ponen en ejecución. Con esta estructura identificada, se realizó un diagrama de clases, identificando los distintos atributos y las funciones utilizadas. Con este diagrama en mente, se estudió cada clase por si sola, explicando la función de cada apartado del código y su aportación global al ejercicio.

A su vez el apartado 7, se debe tratar de manera distinta, pues se trata de tareas dentro de un proyecto externo. Por ello, antes de analizarse el ejercicio se realizó un pequeño estudio del proyecto general. Con el enfoque general en mente, se realizó otro estudio de las tareas realizadas. A diferencia del apartado anterior, este estudio se basará en la depuración realizada del código, en el cual se realizaron una serie de correcciones para su correcto funcionamiento.

El apartado 8, enfocado a la implementación en el robot de políticas, se estudió un caso real de esta implementación \cite{Le_Lay_Kinova_Gen3_RL_2025}, pero en un robot distinto. Estudiado este caso, se diseño un programa para su uso en el \emph{UR3} \emph{Robohabilis} \cite{metatool} \cite{robot_ur3e}, utilizando el lenguaje \emph{URScript} \cite{URScriptLanguage} para el control del Robot y algunas clases implementadas en el caso estudiado.

Por último, una vez terminado el trabajo, se expondrá las conclusiones obtenidas de este, basándose plenamente en la experiencia obtenida en el transcurso del proyecto.

En conclusión, este trabajo sigue distintas metodologías, dependiendo si se trata de un enfoque práctico o teórico. Tomando en conjunto todo, se podría definir una metodología general. Primero, se realiza el estudio teórico, tanto del aprendizaje en refuerzo general como el de la herramienta específica. Después, se utiliza los conocimientos obtenidos para realizar estudios prácticos y analíticos.

Toda la metodología y alcance descrito se concretan en este documento. A continuación, se explicará como se encuentra estructurado.

\section{Estructura del documento}
PENDIENTE
\chapter{Estado del arte del Aprendizaje por Refuerzo (RL) en robótica}

\section{Introducción al RL aplicado a la robótica}
El aprendizaje por refuerzo es una forma de aprendizaje en la que, a través de interactuar con el entorno, se trata de maximizar una recompensa numérica \cite[Pág. 1]{sutton_reinforcement_2020}. Este paradigma se caracteriza por no tener instrucciones definidas sobre cómo actuar y por una realimentación retrasada en el tiempo. 

El primer ejemplo del uso de esta disciplina en la robótica se remonta a 1992, donde métodos de aprendizaje por refuerzo se aplicaron en un robot basado en comportamientos, \emph{Obélix} \cite{mahadevan_automatic_1992}. En este experimento se utilizaba un algoritmo basado en un entorno de aprendizaje por refuerzo para que dicho robot empujase una caja. 

En la actualidad, el aprendizaje por refuerzo está afianzado en la robótica como una disciplina de rápido desarrollo. Especialmente, el aprendizaje por refuerzo profundo, basado en la implementación del aprendizaje por refuerzo para crear redes neuronales profundas \cite{francois-lavet_introduction_2018}, ha obtenido resultados muy relevantes en estados con un gran número de dimensiones o altamente no lineales, donde otros métodos de control prueban ser muy ineficientes. Estos resultados se han mostrado en multitud de disciplinas dentro de este campo, como la locomoción, la navegación o la manipulación. Además, se ha mostrado también su efectividad tanto en robots individuales como colaborativos \cite{tang_deep_2024}.

A continuación, se presentan distintos casos de éxitos para distintas disciplinas.

\section{Aplicaciones en manipulación}
La manipulación se da cuando un robot altera su entorno a través de contacto selectivo \cite{mason_toward_2018}. La manipulación presenta un gran desafío para cualquier método de aprendizaje, debido a la gran cantidad de observaciones y acciones necesarias para llevar a cabo distintas tareas, las cuales pueden llegar a ser bastante elaboradas. Todo esto lleva a un gran coste computacional y a una elevada complejidad a la hora de simular la física y los espacios de trabajo. Además, la transferencia del aprendizaje al mundo real se vuelve lento e inseguro. A pesar de esto, los métodos de aprendizaje por refuerzo profundo han obtenido un notable éxito dentro de esta disciplina \cite{tang_deep_2024}.

Un ejemplo de esta aplicación se da en el artículo \emph{“QT-Opt: Scalable Deep Reinforcement Learning for Vision-Based Robotic Manipulation”} \cite{kalashnikov_qt-opt_2018}. En este trabajo se utilizan métodos de aprendizaje por refuerzo para generalizar el agarre de objetos desconocidos. Para ello, se emplea como entrada una cámara RGB para poder obtener datos acerca de la forma del objeto. A partir de esta entrada, se desarrolla un algoritmo denominado QT-opt para elegir una acción de agarre, conformando una función de valor acción-estado Q y resolviendo esta para obtener el máximo valor de éxito. En el apartado 3, se entrará en detalle sobre cómo se conforma esta función y los distintos algoritmos que se pueden usar para resolverla.

En la actualidad, se encuentran casos como \emph{“DORA: Object Affordance-Guided Reinforcement Learning for Dexterous Robotic Manipulation”} \cite{zhang_dora_2025}. En él, se propone una nueva aplicación de manipulación para el agarre de objetos siguiendo mapas de \emph{affordances}. Los mapas de \emph{affordances} codifican la superficie de un objeto según las regiones funcionales de este. Este mapa se incluye como información adicional al estado del MDP (concepto que se explica en el apartado 3.2). Combinando el RL con estos mapas se obtiene un agarre más funcional, permitiendo agarrar más veces un martillo por su mango.

Además de estos ejemplos de investigación, esta tecnología se ha comenzado a aplicarse en el entorno industrial. Covariant, una empresa dedicada a la implementación de la inteligencia artificial en la robótica \cite{covariant_definition}, ha desarrollado un robot basado en modelos de aprendizaje por refuerzo. Este robot ha sido entrenado mediante datos multimodales e interacciones físicas reales con el objetivo de ejecutar diversas tareas de manipulación. Covariant sostiene que su robot RFM-1 es capaz de realizar tareas de segmentación e identificación a través de imágenes, así como ejecutar agarres a través de instrucciones de texto y observaciones \cite{covariant_aplication}.

\section{Aplicaciones en locomoción}
La locomoción en robótica tiene como objetivo utilizar los motores integrados del robot para transportarse por su entorno. Antes del desarrollo del aprendizaje profundo, la locomoción ya estaba estrechamente ligada a esta disciplina dando grandes avances en el desarrollo de cuadrúpedos. Ya entrada en la era del aprendizaje profundo, se llevó su implementación a otros problemas de locomoción, como robots bípedos \cite{tang_deep_2024}.

Pese a que los primeros ejemplos de RL en locomoción se aplicaron a estos cuadrúpedos, el verdadero avance de estos llegó con la implementación del DRL \cite{tang_deep_2024}. En \emph{RMA: Rapid Motor Adaptation for Legged Robots} \cite{kumar_rma_2021} se propone un método para el control de cuadrúpedos en entornos rocosos. En este ejemplo se implementa el aprendizaje por refuerzo sobre una política adquirida mediante aprendizaje supervisado. De esta manera, mediante el aprendizaje supervisado se busca estimar un vector intrínseco del entorno que detalla sus propiedades. Posteriormente, en la fase de implementación, se aplica un algoritmo de aprendizaje por refuerzo, que recibe este vector como entrada, para adaptarse así al entorno actual. De este modo, se obtiene una gran eficiencia en nuevos entornos. 

La locomoción bípeda, comparada a la locomoción de cuadrúpedos, consta de un problema más complejo. Debido a un menor número de apoyos, se produce una falta de redundancia y una reducción de la estabilidad, haciendo necesario un control más preciso y complejo. Sin embargo, gracias al aprendizaje por refuerzo profundo (DRL) han aparecido casos de éxito en este campo, logrando superar al control clásico en ciertos aspectos. En \emph{Reinforcement Learning for Versatile, Dynamic, and Robust Bipedal Locomotion Control} \cite{li_reinforcement_2024}, se presenta un modelo de aprendizaje para el control de bípedos. En él, proponen un doble representación del estado, combinando un registro a corto plazo con otro a largo. Gracias a esto, se obtiene un control robusto, pudiendo adaptarse a las distintas formas de contacto y los cambios en la estabilidad, manteniendo un proceso de adaptación continuo.

\section{Otras disciplinas}
Las disciplinas de locomoción y manipulación serán las que principalmente se tratan en este trabajo. Sin embargo, no son las únicas para las que esta tecnología ha sido utilizada. En este apartado, se estudiará su aplicación en algunas de estas disciplinas.

La navegación es una de estas disciplinas influenciadas por el aprendizaje por refuerzo. Según el estándar IEEE 172-1983, se define como el proceso de dirigir un vehículo a un destino \cite{noauthor_ieee_1983}. No debe confundirse con la locomoción, que se centra en buscar cómo se debe dar este desplazamiento. En \emph{Socially aware navigation for mobile robots: a survey on deep reinforcement learning approaches} \cite{kabir_socially_2025} se muestran distintos enfoques de la navegación donde se implementan algoritmos de DRL. En él se describe como estos algoritmos permiten integrar una capa social a esta navegación, buscando el confort humano, la percepción social, predicción, etc.

Otra disciplina que se ha visto beneficiada por el aprendizaje por refuerzo es la de los robots aéreos, especialmente los UAVs y sus enjambres. Un UAV, también conocido como \emph{dron}, se define por sus propias siglas \emph{Unmanned Aerial Vehicles}; es decir, son vehículos aéreos controlados que realizan tareas sin operación humana \cite{mohsan_unmanned_2023}. Un enjambre se puede definir como un conjunto de robots moviéndose conjuntamente y con un control compartido, mostrándose una cooperación entre los distintos integrantes del grupo \cite{cheraghi_past_2021}. Tanto la complejidad del control de los UAV como de sus grupos se ha abordado mediante herramientas de aprendizaje por refuerzo. Por ejemplo, en \emph{Application of Deep Reinforcement Learning to UAV Swarming for Ground Surveillance} \cite{arranz_application_2023}, se aplican algoritmos PPO para controlar los UAV, donde cada uno de ellos tiene un sub-agente entrenado con este algoritmo. Este enjambre se diseñó para tareas de vigilancia de áreas, pudiendo buscar y fijar objetivos terrestres.

Por último, pese a no ser una disciplina propia de la robótica, la visión artificial también ha sido influenciada por el RL. Esta suele integrarse dentro de modelos VLA (\emph{Vision-Language-Action}). Estos son sistemas que toman observaciones visuales y instrucciones en lenguaje natural para generar órdenes de control \cite{kawaharazuka_vision-language-action_2025}. Recientemente se han comenzado a introducir enfoques para integrar el RL dentro de estos modelos. Esto mismo se propone en \emph{A Survey on Reinforcement Learning of Vision-Language-Action Models for Robotic Manipulation} \cite{deng_survey_nodate}, así como una evaluación del problema Sim2Real y la exploración segura.

Este último problema mencionado, el Sim2Real, es algo constante en todas estas disciplinas. Hace referencia a la implementación de las políticas en robots reales. Sin embargo, este problema se manifiesta en la parte final del aprendizaje por refuerzo y por ende, se abordará al final de este, en el capítulo \ref{ch:sim2real}.

\section{Conclusiones del estado del arte}

Como se puede observar, el RL es una herramienta ampliamente utilizada en la robótica. El desarrollo del aprendizaje profundo, junto con su capacidad de manejar sistemas no lineales y espacios de estados-acción complejos han permitido la implementación del RL. Cabe resaltar cómo el RL se aplica a disciplinas de diferentes necesidades, adaptándose a la alta dimensionalidad de la manipulación o el estudio dinámico de la locomoción; así como la coordinación multi-agente de los enjambres, la interacción social de la navegación o la combinación de informaciones en el VLA. Todos estos factores lo hacen óptimo para los distintos temas que se tratan en este trabajo. No obstante, antes de comenzar con ejercicios de aprendizaje, se profundiza extensamente en el marco teórico del RL.

\chapter{Fundamentos teóricos del Aprendizaje por Refuerzo}

\section{El Aprendizaje por Refuerzo dentro del Aprendizaje Automático}
El aprendizaje por refuerzo pertenece a una disciplina más grande, el aprendizaje automático. Esta disciplina agrupa todos los ejercicios en los que una máquina aprende acerca de un entorno. Se dice que un programa aprende cuando de una experiencia, respecto a una tarea y una medida de éxito, si su rendimiento en dicha tarea mejora con la experiencia, en función de la medida seleccionada \cite[Pág. 1]{mitchell_machine_1997}.

Esta disciplina tiene tres grandes ramas: el aprendizaje supervisado, el aprendizaje no supervisado y el aprendizaje por refuerzo. 

El aprendizaje supervisado usa pares de entradas y salidas conocidos para generalizar información nueva \cite[Pág. 137]{Goodfellow-et-al-2016}. Estos pares conocidos sirven para organizar futuros pares entrada y salidas. Un problema de aprendizaje supervisado, podría ser identificar tipos de animales mediante una base de datos previa. En este caso, se alimenta al modelo con imágenes de animales (entrada) y su nombre (salida). El modelo deberá crear relaciones entre ambos. Se evalúa finalmente al programa, por su habilidad de identificar imágenes de animales a su nombre.

El aprendizaje no supervisado, por otro lado, utiliza directamente las entradas, sin una salida asociada \cite[pág 740]{russell2021ia}. Esto hace que el programa deba buscar patrones lógicos inherentes a su clasificación. Estos patrones se basan en características a estudiar \cite[Pág. 142]{Goodfellow-et-al-2016}. Un problema de aprendizaje no supervisado podría ser agrupar imágenes de animales en función de su especie. Es este caso, se alimenta al programa solo con las imágenes. Siguiendo únicamente la composición de los animales mostrados, deberá agruparlos. 

La frontera entre ambas disciplinas puede resultar difusa. No existe una diferencia formal entre ambas, pues la diferencia entre una característica a estudiar y una salida asociada no es absoluta \cite[Pág 142]{Goodfellow-et-al-2016}. El aprendizaje por refuerzo se diferencia de ambas a través de una única señal de realimentación (acorde a la definición presentada en el apartado 2.1.). De este modo, esta disciplina combina la supervisión del aprendizaje supervisado, con la ventaja de no requerir una gran base de datos catalogada. Gracias a esto, se puede realizar un aprendizaje secuencial sin disponer de un modelo del entorno, lo que la hace especialmente útil para el estudio de la robótica.

\section{Estructura del Aprendizaje por Refuerzo}

El aprendizaje por refuerzo esta definido por una estructura básica. Esta estructura viene de la formalización del problema como un Proceso de Decisión Markov (MDP) \cite[Pág. 47]{sutton_reinforcement_2020}. Esto proviene de la propia naturaleza del problema, por lo que existe ligada al Aprendizaje por Refuerzo. En el próximo apartado, se estudiarán a fondo los MDPs. Sin embargo, al ser la estructura la base de esta disciplina, se presenta primero.

La estructura del aprendizaje por refuerzo define las interacciones entre un agente y un entorno. El agente ejerce acciones sobre el entorno, influyendo en el activamente. El entorno aporta al agente observaciones y recompensas, obteniendo así información sobre él. Además, el entorno, tiene asociado un estado. \ref{fig:agente_entorno}

\begin{figure}[h!]
    \centering
    \includegraphics[width=0.8\textwidth]{imagenes/StructRL.pdf}
    \caption{Interacción agente–entorno.}
    \label{fig:agente_entorno}
\end{figure}
\chapter{Análisis de la herramienta IsaacLab}

En esta capítulo se va a introducir la herramienta IsaacLab. Haciendo uso de la documentación oficial de IsaacLab \cite{isaaclab_docs}, el informe técnico \cite{mittal2025isaaclab} y la biblioteca de APIs \cite{isaaclab_api}. El objetivo es obtener una visión global de cómo funciona la herramienta y cómo generar episodios, para luego después tener la capacidad de entender y crear entornos. En este capítulo se estudiarán los conceptos básicos, los cuales se estudiaran a fondo en los próximos capítulos con ejemplos prácticos.

\section{¿Qué es IsaacLab?}%Lunes
IsaacLab es un módulo de trabajo para el entrenamiento de robots. Su principal objetivo es simplificar las rutas de trabajo en este tipo de ejercicios \cite{isaaclab_doc}. IsaacLab esta enfocado en el trabajo sobre GPUs, combinando la renderización de imágenes realistas con el motor de físicas \emph{PhysX} para construir simulaciones fieles a la realidad \cite{mittal2025isaaclab}.

IsaacLab está construido sobre IsaacSim. IsaacSim es una aplicación construida sobre NVIDIA Omniverse, la cual permite desarrollar, simular y probar robots controlados por IA en entornos virtuales \cite{Nvidia_isaac_nodate}. IsaacLab se puede entender como un conjunto de herramientas para usar dentro del simulador IsaacSim. De este modo, pese a que se trabajará enteramente con IsaacLab, se adquirirá en este capítulo algunos conceptos de IsaacSim.

Los principales incentivos para usar IsaacLab son  \cite{isaaclab_doc}:
\begin{itemize}
    \item Modularidad: capacidad de modificar y añadir nuevos entornos, robots y sensores; pudiendo utilizar todos estos en bibliotecas comunes, limitando las modificaciones.
    \item Código abierto: mantenimiento de un código abierto y libre para la comunidad. Esto permite completa libertad para modificar cualquier código y adaptarlo a las necesidades del entorno.
    \item Gran cantidad de ejemplos y recursos: IsaacLab cuenta con un gran número de entornos, sensores y tareas preparadas para el entrenamiento. Esto permite partir de una base sobre la que construir las tareas personalizadas.
\end{itemize}

Por estos motivos, se ha escogido esta herramienta para realizar los entrenamientos. La principal desventaja de esta herramienta es la necesidad de utilizar un hardware específico, las tarjetas RTX de Nvidia. Sin ellas, no se puede utilizar esta herramienta, ya que IsaacLab esta preparado para utilizarlas directamente. Esto saca el máximo partido a las tarjetas gráficas, pero limita el uso a la disposición de estos recursos. Por la parte de este proyecto, se dispuso de estas tarjetas gracias a la cesión de un ordenador por parte del equipo MetaTool. Aprovecho este momento para dar las gracias a Virgilio Gómez Lambo, tanto por los recursos prestados como por su ayuda en el entendimiento de la herramienta y el RL.

En conclusión, IsaacLab es una herramienta para realizar ejercicios de aprendizaje por refuerzo en IsaacSim. Con esto en mente, se va estudiar cual es la estructura externa de la herramienta.

\section{Estrutura de la herramienta}%Lunes
La herramienta IsaacLab se centra en la construcción de los entornos, los cuales luego se someten al aprendizaje. Estos entornos, reciben las acciones del agente (la red neuronal) y procesando las recompensas y observaciones correspondientes \cite[Walkthrough, Enviroment Background Design]{isaaclab_doc}. Para la construcción de estos entornos, IsaacLab utiliza la misma estructura que IsaacSim. Esta estructura define y gestiona los entornos. Se puede imaginar esta estructura como una muñeca rusa, donde cada nivel contiene al resto (figura \ref{fig:lab_struct}).
\begin{figure}[h!]
    \centering
    \includegraphics[width=0.8\textwidth]{imagenes/labStruct.png}
    \caption{Estructura de la herramienta IsaacLab \cite{isaaclab_doc}.}
    \label{fig:lab_struct}
\end{figure}

El primer nivel consiste de la aplicación. La aplicación gestiona los recursos del sistema y es la encargada de lanzar y destruir la simulación \cite{isaaclab_doc}. La aplicación se puede gestionar a partir de la API \emph{isaaclab.app} \cite{isaaclab_api}. Esta API se encarga de gestionar el lanzamiento de la aplicación, así como los distintos argumentos pertinentes a esta.

Esta aplicación, contiene la simulación, la cual, como ya se ha mencionado, es creada y destruida por esta. La simulación es la encargada de definir como funcionará las físicas, el tiempo y la gravedad. La simulación divide el ejercicio en múltiples instantes de tiempo, dividiendo las tareas de cálculo en subprocesos \cite{isaaclab_doc}. La simulación, al igual que la aplicación, tiene su propia API, \emph{isaaclab.sim} \cite{isaaclab_api}. Con esta, se definen parámetros como el tamaño de paso o la fuerza de la gravedad.

La simulación a la vez contiene todos los elementos relevantes a esta, los cuales agrupamos en el concepto de \emph{mundo} \cite{isaaclab_doc}. Este mundo se define por el origen de coordenadas, el cual se toma como referencia para ubicar el resto de elementos. El mundo se estructura en dos elementos más: el escenario y la escena. El escenario, por un lado, 
provee de un contexto geográfico dentro de la escena \cite{isaaclab_doc}. Es decir, permite utilizar dentro de las escenas un origen de coordenadas propio. La escena, por su parte, será la encargada de administrar los elementos que conforman el entorno. 

El entorno como tal, estará organizado por tanto en un escenario (figura \ref{fig:lab_escena}) y administrado por la escena. Es por esto que la escena tiene su propia API, \emph{isaaclab.scene} \cite{isaaclab_api}, la cual nos permite gestionar y obtener datos de todos los elementos del entorno. Estos elementos se organizan en \emph{primarios}, elementos separados dentro de la organización del escenario que son importados a la escena a través de un archivo USD (figura \ref{lab_escena}) \cite{isaaclab_doc}. USD, por su propia parte, es el lenguaje de descripción robots y entornos \cite{isaaclab_usd}. En este trabajo no manejaremos este lenguaje, pero si utilizaremos los ficheros de este tipo para importar los primarios.
\begin{figure}[h!]
    \centering
    \includegraphics[width=0.8\textwidth]{imagenes/labScene.png}
    \caption{Organización de un escenario dentro de IsaacLab \cite{isaaclab_doc}.}
    \label{fig:lab_escena}
\end{figure}

IsaacLab entonces esta organizado en 5 conceptos fundamentales. Se asienta una referencia central, el mundo. Este mundo contiene el escenario y la escena, los cuales organizan los entornos. El mundo está contenido por la simulación, la cual define las propiedades de este. Por último, la simulación es gestionada por la aplicación. Esta es la estructura externa a los entornos, sobre la cual se asientan. A continuación, se estudiará que maneras hay de construir los entornos.

\section{Arquitectura de entornos}
Teniendo en cuenta la estructura anterior, se procede ahora a estudiar la construcción de entornos. Se recuerda que los entornos son los encargados de recoger las acciones del cliente y procesar las observaciones y recompensas. A parte de esta definición, se debe tener en cuenta el campo en el que se trabaja, la robótica. Por esto, el eje central de todos los entornos será el robot. Con todo esto en mente, se pueden definir los objetivos del diseño de entornos \cite{isaaclab_doc}:
\begin{enumerate}
    \item Definir el robot y las acciones.
    \item Definir los parámetros de la simulación.
    \item Definir la forma de clonado y el número de entornos.
    \item Calcular y entregar las acciones y recompensas
    \item Definir los estados absorbentes y los reinicios.
\end{enumerate}

Los entornos estarán constituidos por un robot y el resto de su entorno (objetos, obstáculos, efectos visuales, etc.). Sobre este entorno se definirán las acciones (asociadas directamente al robot), las recompensas y las observaciones. También deberemos definir dentro de este entorno los estados absorbentes (finales). En algunas ocasiones el robot alcanzará un estado donde no será relevant continuar con el aprendizaje. En ese estado, se cerrará el episodio y se reiniciará el entorno. Por último, IsaacLab nos permite optimizar el tiempo de entrenamiento clonado los entornos. De este modo, en vez de tener un único entorno, se pueden tener múltiples entornos entrenando simultáneamente.

Para definir todos estos aspectos, existen dos principales maneras de programar los entornos:

\subsection{Direct Based}%Miércoles
La manera directa, como su propio nombre indica, es la más franca de las dos. Esta forma permite implementar todos los puntos antes mencionados en un mismo \emph{script}. Los entornos directos heredan de una clase \emph{DirectRLEnv}, dentro de la API \emph{isaaclab.envs}. Para programar el entorno entonces, se definen las funciones abstractas de esta clase. Seguidamente, la clase se envuelve en un \emph{wrapper} y se alimenta a una de las bibliotecas con los algoritmos de aprendizaje por refuerzo. Este proceso se verá en detalle en el apartado \ref{ap:entrenamiento}. A su vez, un ejemplo de este tipo de construcción se verá en el capítulo \ref{ch:araña}.
 
En la figura \ref{fig:lab_scheme_direct} se puede ver un esquema de las interacciones de la clase \emph{DirectRLEnv}, bajo el nombre \emph{Enviroment Scripting}. Esta clase se encarga de comunicar el entrono (la escena) y el agente. Por tanto, volviendo a la estructura del aprendizaje por refuerzo (\ref{ap:StrAR}), esta clase representaría las interacciones entre ambos elementos.
\begin{figure}[h!]
    \centering
    \includegraphics[width=0.8\textwidth]{imagenes/labScene.png}
    \caption{Organización de un escenario dentro de IsaacLab \cite{isaaclab_doc}.}
    \label{fig:lab_scheme_direct}
\end{figure}


\subsection{Manager Based}%Miércoles


\section{Estructuras de datos}%Viernes
\subsection{Clases}
\subsection{Tensores: PyTorch}

\section{Entrenamiento de agentes}%Sábado
\label{ap:entrenamiento}
\section{Evaluación de agentes}%Sábado
\section{Análisis Global}%Sábado


\chapter{Estudio caso locomoción}
\label{ch:arana}
En este capítulo, se va a estudiar un ejemplo de la herramienta IsaacLab. Con este estudio se pretende analizar las distintas partes de la construcción de entornos a través de la forma directa. Primero, se analizará el caso y el objetivo de este. Después, se realizará un diagrama de clases con las principales clases y sus métodos y atributos más relevantes. Una vez definido el diagrama de clases, se analizará cada una detenidamente, entrando en detalle sobre sus métodos y atributos; se verá la función y definición de cada uno. A continuación, se estudiará el registro a través de \emph{gymnasium}, repasando a su vez la configuración del agente. Registrado el entorno, se procederá al entrenamiento de este y a la evaluación del resultado final. Por último, se propondrán algunas mejoras para futuros estudios de aprendizaje.

\section{Descripción caso práctivo}
El primer ejemplo escogido para el estudio es el entorno "Isaac-Ant-v0". En este entorno se busca enseñar a andar a un robot araña de cuatro patas, en IsaacLab llamado \emph{Ant} (figura \ref{fig:antrobot}). El objetivo principal será desplazar el robot en un dirección concreta, manteniendo el torso paralelo al suelo. Se considerará que el robot va paralelo al suelo cuando su plano en x e y sea paralelo al plano x e y del origen de coordenadas.
\begin{figure}[ht]
    \label{fig:antrobot}
    \centering
    \includegraphics[width=\linewidth]{imagenes/antrobot.jpg}
    \caption{Robot araña o \emph{Ant}, objetivo del aprendizaje para el primer caso práctico.}
\end{figure}

Analizar este ejercicio es una parte integral de este trabajo. El objetivo a futuro de este trabajo es crear una guía para realizar futuros ensayos de aprendizaje por refuerzo. Este caso, se relaciona directamente con dos proyectos internos de la universidad, \emph{Romerín} \cite{Romerin_Descrip} y \emph{Tarántula} (en fase de desarrollo). Por tanto, este análisis tiene dos objetivos: analizar el problema concreto de locomoción para robots araña y estudiar un caso práctico de la programación directa.

El código de este ejercicio se ha extraído de la herramienta IsaacLab; este se puede encontrar dentro del repositorio de la herramienta \cite{mittal2025isaaclab}, accesible desde la documentación \cite{isaaclab_doc}. En este capítulo, se analizará el código desde el diagrama de clases; aportando donde sea necesario los fragmentos de código relevante. Durante el análisis, también se irá indicando donde se encuentra la parte del código a la cual se hace referencia. Se ha preparado un proyecto de IsaacLab con todos los códigos utilizados; por lo cual, se indicará la referencia del código de IsaacLab y el proyecto. Se procederá ahora a la definición del diagrama de clases y su estudio.

\section{Diagrama de Clases}

El diagrama de clases del entorno de la araña se muestra en la figura \ref{UMLarana}. El diagram se utiliza para obtener una visión general de el código del entorno y para simplificar el futuro análisis de este. No se incluyen la totalidad de métodos y atributos, pues gran parte de estos no son relevantes para casos generales como los que se estudiarán. El resto de métodos y atributos son menos relevantes, usándose para funcionalidades muy concretas o para procesos internos de IsaacLab.

El diagrama muestra la construcción del entorno, sobre el cual se entrenará en el apartado \ref{ap:regisarana} y \ref{ap:entrearana}. El entorno gira alrededor de dos piezas centrales, la clase \clase{AntEnv} y la clase \clase{AntEnvCfg}. Al estar trabajando en el caso directo, ambas clases heredan de sus contrapartes del modo directo: \clase{DirectRLEnv} y \clase{DirectRLCfg}, respectivamente. Estas clases están definidas dentro del código de IsaacLab; cada vez que se construya en un entorno en modo directo se heredara de ambas. 

En el caso del directo, la clase de configuración, aquella que hereda de \clase{DirectRLEnv\allowbreak Cfg}, se encarga de definir los parámetros físicos y de las interacciones del entorno, las características de la simulación y la escena (con el robot y el resto de elementos incluidos). Esta clase, siempre será un atributo de la clase principal del entorno, aquella que hereda de \clase{DirectRLEnv}. Esta segunda clase toma un gran protagonismo en el modo directo. Sobre ella cae la responsabilidad de definir como se implementa la configuración del entorno, definiendo las interacciones y parte del proceso de aprendizaje y creando la escena a partir de lo definido.

Por otro lado, en este caso particular, se debe analizar de donde provienen ambas clases. Por un lado, la clase de configuración \clase{AntEnvCfg}, hereda directamente de la clase de configuración original. Sin embargo, la clase principal del entorno de la araña hereda en un paso previo de una clase \clase{LocomotionEnv}; esta hereda, esta vez sí, de \clase{DirectRLEnv}. Esta clase intermedia es de gran utilidad, ya que generaliza una tarea concreta encargada de resolver el problema de locomoción. De esta manera, se puede heredar de esta clase para cualquier problema de locomoción, ajustando la escena al caso concreto dentro de la configuración y ajustando parámetros concretos en la principal.

Este esquema se repite en la gran mayoría de los casos de programación directa. Por esto, es importante comprender como se implementa y define cada clase. En el próximo apartado, se estudiará detenidamente cada una de las clases.

\begin{landscape}
\begin{figure}[ht]
    \label{UMLarana}
    \centering
    \includegraphics[width=\linewidth]{imagenes/UMLarana.pdf}
    \caption{Diagrama UML del ejemplo araña, programación directa.}
\end{figure}
\end{landscape}

\section{Análisis de clases}

En este apartado se estudiará cada una de las clases mostradas en el diagrama, analizando los métodos y atributos definidos en el diagrama. Para cada una de las clases se indicará donde se puede encontrar el código. Después se explicarán la funcionalidad del método o atributo. Dentro de esta explicación, se mostrarán algunas partes del código donde exista un interés en la implementación del método; especialmente en aquellos que definan observaciones u recompensas del entorno. Se comenzará estudiando la clase principal padre, para pasar después a las distintas clases de configuración y se terminará con las clases heredadas de la primera. 

\subsection{DirectRLEnv}
\label{ap:DirectRLEnv}
\begin{figure}[ht]
    \label{fig:clasedirectrlenv}
    \centering
    \includegraphics[width=0.8\textwidth]{imagenes/ClassDirectRLEnv.png}
    \caption{Imagen del diagrama referente a la clase \clase{DirectRLEnv}.}
\end{figure}

La clase \clase{DirectRLEnc} (figura \ref{fig:clasedirectrlenv}) se encuentra definida en el código fuente de IsaacLab. Se puede acceder al código a través de la biblioteca de API de IsaacLab \cite{isaaclab_api}, concretamente en \emph{isaaclab.envs.DirectRLEnv}. Una vez ahí, se debe seguir el enlace asociado al título, en el botón de "[source]"; tal y como se indica en la figura \ref{fig:guiasource}.


\begin{figure}[ht]
    \label{fig:guiasource}
    \centering
    \includegraphics[width=0.8\textwidth]{imagenes/guiasource.png}
    \caption{Imagen de la documentación oficial de IsaacLab con el link al código fuente \cite{isaaclab_api}.}
\end{figure}

Esta clase, cómo se viene comentando, es el pilar fundamental del entorno. Esta clase, a través de sus métodos crea el entorno y define sus propiedades e interacciones.

El primer elemento relevante de esta clase se trata del atributo definido como \atributo{cfg}. Este atributo almacena una clase \clase{DirectRLEnvCfg}. Este atributo se utiliza constantemente en el resto de la clase, ya que es la configuración del entorno que se pretende construir. Es por esto, que se debe recoger en el constructor, el primer método definido en el diagrama. El constructor de esta clase es complejo y amplio, pero para el enfoque de este trabajo solo se tendrá en cuenta la recepción del atributo \atributo{cfg}. El resto de código va enfocado al propio funcionamiento de IsaacLab, el cual no se estudiará.

El resto de métodos no son definidos en esta clase, sino que son meramente declarados. Exceptuando el método \metodo{\_set\_up\_scene(self)}, el resto serán métodos abstractos. Estos métodos se definen en las clases heredadas, con el objetivo de definir el funcionamiento de la clase. Más adelante, en el sub-apartado (figura \ref{ap:locomotionenv}), se verán ejemplos de sus implementaciones. En este apartado, se estudiará únicamente el objetivo principal de cada una:

\begin{itemize}
    \item \metodo{\_set\_up\_scene(self)}: Se encarga de configurar la escena, implementando los elementos definidos en el configurador.
    \item \metodo{\_pre\_physics\_step(self)}: Define las acciones previas a realizar el cálculo de las físicas del entorno.
    \item \metodo{\_apply\_actions(self)}: En este método se procesan las acciones y se envían al robot entrenado.
    \item \metodo{\_get\_observations(self)}: Se encarga de calcular y definir las observaciones realizadas sobre el entorno.
    \item \metodo{\_get\_rewards(self)}: Este método calcula y define las recompensas obtenidas del entorno. 
    \item \metodo{\_get\_dones(self)}: Este método define y comprueba las condiciones de reinicio del entorno.
    \item \metodo{\_set\_debug\_vis\_impl}: Se encarga de crear o configurar la visualización de los objetos en escena.
\end{itemize}

Esta clase, por tanto, define todas las funciones que deben utilizarse para crear y administrar el entorno. Dentro de esta clase, existen otros métodos como \metodo{step(self)} o \metodo{render(self)}, los cuales utilizan estos métodos para crear el proceso de comunicación con el entorno. Esta parte del código, no es relevante para este trabajo, pues forma parte del funcionamiento propio IsaacLab y no se deberá modificar a la hora de crear los entornos. Cabe resaltar que, a pesar de no ser parte del enfoque del trabajo, para tareas de depuración se ha necesitado comprender este proceso.

Como ya se ha mencionado, el elemento que definirá gran parte de esta implementación sera la clase de configuración. A continuación, se estudiará la clase base para luego analizar las respectivas clases heredadas.

\subsection{DirectRLEnvCfg}

\begin{figure}[ht]
    \label{fig:clasedirectrlenvcfg}
    \centering
    \includegraphics[width=0.5\textwidth]{imagenes/ClassDirectRLEnvCfg.png}
    \caption{Imagen del diagrama referente a la clase \clase{DirectRLEnvCfg}.}
\end{figure}

La clase \clase{DirectRLEnvCfg} (figura \ref{fig:clasedirectrlenvcfg}), al igual que la anterior, se encuentra definida el código fuente; pudiéndose acceder de la misma manera desde la API \api{isaaclab.evns.DirectRLEnvCfg}. Esta clase esta definida como una \emph{config\_class}. Este tipo de clase se introdujo en el apartado \ref{ap:clasesconfigclass}. Esta tipo de clase almacena únicamente atributos, haciéndola más fácil de gestionar dentro del funcionamiento de la herramienta. En este apartado, se van a enumerar y analizar los atributos más relevantes de esta clase y cómo afectan al entorno.
\begin{itemize}
    \item \atributo{sim}: Almacena una clase \clase{SimulationCfg}, encargada de configurar los principales parámetros de la simulación.
    \item \atributo{decimation}: Amacena un valor numérico entero (int) que define el número de acciones realizadas antes de actualizar la política.
    \item \atributo{episode\_length\_s}: Almacena un valor numérico decimal (float) que define la duración de un episodio.
    \item \atributo{scene}: Almacena una clase \clase{InteractiveSceneCfg} que define los elementos incluidos dentro de una escena, así como las propiedades de esta.
    \item \atributo{obs\_space}: Almacena una clase \clase{SpaceType} que indica el número de observaciones realizadas sobre el entorno.
    \item \atributo{action\_space}: De igual manera que el anterior, almacena una clase \clase{SpaceType} que indica el número de acciones.
\end{itemize}

Cabe resaltar un par de cosas acerca de estos atributos. Exceptuando el atributo \atributo{sim}, el resto tienen asociada una constante \atributo{MISSING}. Esta constante se asegura de que estos atributos sean definidos dentro de una posible clase heredada; es decir, todos los atributos deberán ser definidos en una clase específica de configuración. En segundo lugar, es interesante notar que existen dos variables para el número de las acciones y las observaciones pero no para las recompensas. Esto es debido a que la recompensa deberá definirse como una señal numérica, tal y como dicta el aprendizaje por refuerzo. El tamaño de las observaciones y las acciones por su parte definirán la dimensión de la red neuronal.

Vista la clase base de la configuración de entorno, se va estudiar como se hereda de ella para comenzar a definir un entorno concreto.

\subsection{AntEnvCfg}

\begin{figure}[ht]
    \label{fig:antenvcfg}
    \centering
    \includegraphics[width=0.5\textwidth]{imagenes/AntEnvCfg.png}
    \caption{Imagen del diagrama referente a la clase \clase{AntEnvCfg}.}
\end{figure}

La clase a estudiar, es la clase \clase{AntEnvCfg}. Esta clase esta definida dentro del repositorio IsaacLab, en el directorio \verb|source/isaaclab_tasks/isaaclab_tasks/direct/ant/| \verb|ant_env.py|
\cite{mittal2025isaaclab}. Esta clase hereda directamente de la clase \clase{DirectRLEnvCfg}, incluyendo dos nuevos atributos: \atributo{terrain} y \atributo{robot}. Por su parte, \atributo{terrain} almacena una clase \clase{TerrainImporterCfg}, encargada de configurar el terreno del entorno. Por otro lado, el atributo \atributo{robot} se encarga de definir las características del robot a entrenar, almacenando una clase \clase{ArticulationCfg}; esta clase se implementará en el siguiente apartado.

En este caso, al ser una implementación de una clase, se va estudiar el código detenidamente.

Al comienzo del código, se importan las distintas herramientas y bibliotecas que vamos a utilizar. Entre ellas se pueden encontrar las clases de simulación, las clases de configuración, etc. Para poder importar una clase, un método o una constante, primero se debe localizar la api donde esta definida y después indicarla. En el código \ref{lst:impantapi}, se puede ver un ejemplo, donde se importa la clase \clase{TerraimImporterCfg} de la API \api{isaaclab.terrains}. También se pueden importar clases definidas en archivos aparte, como se hace con la constante \atributo{ANT\_CFG} (código \ref{lst:impantdir}), que guarda la configuración del robot.

\begin{lstlisting}[style=mypython, caption={Ejemplo para importar una clase de una API},  label={lst:impantapi}]
from isaaclab.terrain import TerrainImporterCfg
\end{lstlisting}

\begin{lstlisting}[style=mypython, caption={Ejemplo para importar una clase de un archivo},  label={lst:impantdir}]
from isaaclab_assets.robots.ant import ANT_CFG
\end{lstlisting}

Seguidamente, se comienza a definir la clase. En primer lugar, se definen distintos atributos concretos. Entre ellos se encuentran los ya mencionados \atributo{episode\_length\_s}, \atributo{action\_scale}, \atributo{decimation} y \atributo{observation\_space}. También se definen algunos nuevos atributos, como \atributo{action\_scale}, que sirve para escalar la acción en el procesado. Seguidamente se configura la simulación (código \ref{lst:simcfgant}). En el constructor, se definen dos atributos principales: \atributo{dt}, que define el tiempo entre los pasos del proceso, y \atributo{render\_interval}, que define cada cuanto se actualiza la visualización.Después se define el atributo terrain, con una clase \clase{TerrainImporter}. Este atributo define cómo será el suelo, desde su construcción hasta sus propiedades físicas. En este caso, no cabe resaltarlo pues se genera un plano simple, pero en el apartado de mejoras, se estudiará detenidamente esta clase para generar otro tipo de terrenos.
\begin{lstlisting}[style=mypython, caption={Definición de la configuración de la simulación},  label={lst:simcfgant}]
sim: SimulationCfg = SimulationCfg(dt=1 / 120, render_interval=decimation)
\end{lstlisting}

Continuando dentro de la clase, se define el atributo \atributo{scene}, mediante una clase \clase{InteractiveSceneCfg} (código \ref{lst:scenecfgant}). Dentro de esta clase, se definen con el constructor distintos parámetros referentes al número de entornos. Como ya se ha mencionado, en IsaacLab se entrena con múltiples copias de un mismo entorno en paralelo. Esta clase es la encargada de almacenarlos y gestionarlos. Por ello, se deben definir algunos parámetros relevantes como el número de entornos (\atributo{num\_envs}), el espacio entre estos (\atributo{env\_spacing}) y la forma de clonado (\atributo{replicate\_physics} y \atributo{clone\_fabric}). Justo después, se define el atributo \atributo{robot}, encargado de configurar el robot del entorno. Este atributo se asocia a una constante, importada, como antes se ha visto, de un archivo a parte. En el próximo apartado se verá como se configura el robot araña. También se define el atributo \atributo{joint\_gears}, encargado de ajustar la fuerza aplicada en las acciones. Estas acciones también van estrechamente relacionadas con la configuración del robot, configuradas también en la constante importada.
\begin{lstlisting}[style=mypython, caption={Definición de la configuración de la escena},  label={lst:scenecfgant}]
scene: InteractiveSceneCfg = InteractiveSceneCfg(
    num_envs=4096, env_spacing=4.0, replicate_physics=True, clone_in_fabric=True)
\end{lstlisting}

Por último, para terminar de definir la configuración del entorno, se deben indicar los pesos que se van a utilizar para cada recompensa. En el apartado \ref{ap:locomotionenv} se verá cuales son estas recompensas y como se aplica este peso. No obstante, antes de llegar a estas se va estudiar la configuración del robot.

\subsection{ArticulationCfg}

\begin{figure}[ht]
    \label{fig:clasearticulationcfg}
    \centering
    \includegraphics[width=0.5\textwidth]{imagenes/ArticulationCfg.png}
    \caption{Imagen del diagrama referente a la clase \clase{ArticulationCfg}}
\end{figure}

La clase \clase{ArticulationCfg} sirve para configurar la implementación del robot del entorno. Esta configuración se puede definir a través de su constructor. En este apartado, se estudiará la implementación de esta clase para el caso concreto de locomoción para el robot araña. Esta implementación se realiza en el archivo 
\verb|source/isaaclab_assets/| \verb|isaaclab_assets/robots/ant.py|
 \cite{mittal2025isaaclab}, cuyo código se muestra en \ref{lst:artcfgant}.

Esta clase hereda de la llamada \clase{AssetBaseCfg}, dirigida a configurar cada prim de la simulación. Esta clase, asocia el prim a una dirección dentro del mundo (definido en el apartado \ref{ap:structisaac}) y define la forma en la que se crea, normalmente a través de un archivo USD. También define si este archivo es visible, a través del atributo \atributo{debug\_bis} y con que objetos puede colisionar, mediante el atributo \atributo{colission\_group}. Por esto, usaremos estos mismos atributos para definir el robot.

\begin{lstlisting}[style=mypython, caption={Implementación de la clase \clase{ArticulationCfg}},  label={lst:artcfgant}]
from __future__ import annotations

import isaaclab.sim as sim_utils
from isaaclab.actuators import ImplicitActuatorCfg
from isaaclab.assets import ArticulationCfg
from isaaclab.utils.assets import ISAAC_NUCLEUS_DIR

ANT_CFG = ArticulationCfg(
    prim_path="{ENV_REGEX_NS}/Robot",
    spawn=sim_utils.UsdFileCfg(
        usd_path=f"{ISAAC_NUCLEUS_DIR}/Robots/IsaacSim/Ant/ant_instanceable.usd",
        rigid_props=sim_utils.RigidBodyPropertiesCfg(
            disable_gravity=False,
            max_depenetration_velocity=10.0,
            enable_gyroscopic_forces=True,
        ),
        articulation_props=sim_utils.ArticulationRootPropertiesCfg(
            enabled_self_collisions=False,
            solver_position_iteration_count=4,
            solver_velocity_iteration_count=0,
            sleep_threshold=0.005,
            stabilization_threshold=0.001,
        ),
        copy_from_source=False,
    ),
    init_state=ArticulationCfg.InitialStateCfg(
        pos=(0.0, 0.0, 0.5),
        joint_pos={
            ".*_leg": 0.0,
            "front_left_foot": 0.785398,  # 45 degrees
            "front_right_foot": -0.785398,
            "left_back_foot": -0.785398,
            "right_back_foot": 0.785398,
        },
    ),
    actuators={
        "body": ImplicitActuatorCfg(
            joint_names_expr=[".*"],
            stiffness=0.0,
            damping=0.0,
        ),
    },
)
\end{lstlisting}

Esta implementación se almacena en la constante \atributo{ANT\_CFG}, que luego se importa, como ya se ha visto en el apartado anterior, dentro de  la configuración del entorno. En el constructor, primero se definen los dos atributos heredados de la clase \clase{AssetBaseCfg}. 

En primer lugar, el atributo \atributo{prim\_path}, el cual define la ruta donde se guarda el elemento primitivo. Este atributo usa una cadena formateada que permite almacenarlo en cada uno de los entornos, manteniendo el mismo esquema. En 

Segundo lugar, el atributo \atributo{spawn}, que define la creación del primitivo. Este atributo se define a través de una clase \clase{UsdFileCfg}. Esta clase indica el archivo que se utiliza para generar el robot en la escena, mediante el atributo \atributo{usd\_path}. Este archivo se encuentra guardado dentro de IsaacSim, por lo que se usa la constante \atributo{ISAAC\_NUCLEUS\_DIR}, que apunta a los archivos de esta aplicación. También se definen las propiedades relevantes a la articulación con los atributos \atributo{rigid\_props} y \atributo{articulation\_props}. Por último, mediante el atributo \atributo{copy\_from\_source}, se indica si se usará una copia del archivo o el propio archivo. En este caso, al no realizar modificaciones, se indica con un valor \atributo{False} el uso del archivo original.

Los otros dos atributos que se deben indicar en el constructor son \atributo{init\_state} y \atributo{actuators}. Por un lado, \atributo{init\_state} define la posición inicial del robot mediante la clase \clase{InitialStateCfg}. En el constructor de esta clase, se debe indicar la posición del robot referente al mundo, mediante el atributo \atributo{pos}; y la posición de las articulaciones. La posición de las articulaciones se indica mediante un diccionario. En él, a todas las patas se les asocia el mismo valor, utilizando una cadena con el caracter "*". Esto hace que todas las articulaciones terminadas en "\_leg" se les asocie el mismo valor. Por otro lado, el atributo \atributo{actuators} define el movimiento de las articulaciones, definiéndose a través de un diccionario. En este caso, se define un único tipo de movimiento mediante \clase{ImplicitActuatorCfg}, en la cual se asocia el movimiento a todas las articulaciones y se dan los valores de rigidez (\atributo{stiffness}) y amortiguación (\atributo{damping}).

Definida con esta clase el robot, se tienen todos los elementos necesarios para construir el entorno. En el siguiente apartado, se estudiará la clase \clase{LocomotionEnv}, que hereda de \clase{DirectRLEnv} y define los entornos e interacciones de las tareas de locomoción.

\subsection{LocomotionEnv}
\label{ap:locomotionenv}

\begin{figure}[ht]
    \label{fig:locomotionenv}
    \centering
    \includegraphics[width=0.5\textwidth]{imagenes/LocomotionEnv.png}
    \caption{Imagen del diagrama referente a la clase \clase{LocomotionEnv}.}
\end{figure}

En este apartado se va estudiar la definición de la clase \clase{LocomotionEnv}. Al ser el elemento principal que describirá la tarea se van a analizar cada uno de sus métodos, viendo tanto su objetivo como el desarrollo del código. Para ello, se expondrá el método y su objetivo, después se mostrará el código y se explicará el contenido. Todo el código de la clase se encuentra en el repositorio de la herramienta IsaacLab, en el directorio \verb|source/isaaclab_tasks/isaaclab_tasks/direct/locomotion/locomotion_env.py|.

El primer método implementado en el archivo es \metodo{normalize\_angle(x)} (código \ref{lst:normang}). Este método se encarga simplemente de utilizar herramientas de PyTorch para normalizar el ángulo. Con esta función, los ángulos se traspasan a un rango $[-\pi, \pi]$. Esto convierte a los ángulos en números más fáciles de tratar, pues se evita el uso de números mayores con el incremento por vuelta.
\begin{lstlisting}[style=mypython, caption={Definición del método \metodo{normalize\_angle(x)}},  label={lst:normang}]
def normalize_angle(x):
    return torch.atan2(torch.sin(x), torch.cos(x))
\end{lstlisting}
Este método se implementa de manera sencilla, utilizando la función de PyTorch \metodo{torch.\allowbreak atan2(y, x)} \cite{pytorch_docs}, la cual devuelve un valor en el rango estipulado. Esta función es alimentada con otras dos funciones de esta biblioteca \metodo{torch.sin(x)} y \metodo{torch.cos(x)} \cite{pytorch_docs}. Estas hacen que se preserve la dirección angular y se elimine el número de vueltas. Como se verá en el resto de métodos, se van a utilizar multiples funciones de PyTorch, ya que en todo momento se trabaja con tensores; estos, como ya se ha comentado, permiten almacenar la información de todos los entornos en un único lugar. Estos métodos se almacenan en el código mediante el objeto módulo \api{torch}.

Definida esta función, que será de utilidad en próximos métodos, se declara la clase \clase{LocomotionEnv}. El primer método definido en esta es su constructor. El constructor busca almacenar todos los datos relevantes que se conocen de primera mano, así como declarar tensores relevantes para el cálculo de las recompensas y observaciones. A continuación, se expone el código de este método (código \ref{lst:initloc})
\begin{lstlisting}[style=mypython, caption={Definición del constructor de la clase \clase{LocomotionEnv}},  label={lst:initloc}]
def __init__(self, cfg: DirectRLEnvCfg, render_mode: str | None = None, **kwargs):
        super().__init__(cfg, render_mode, **kwargs)

        self.action_scale = self.cfg.action_scale
        self.joint_gears = torch.tensor(self.cfg.joint_gears, dtype=torch.float32, device=self.sim.device)
        self.motor_effort_ratio = torch.ones_like(self.joint_gears, device=self.sim.device)
        self._joint_dof_idx, _ = self.robot.find_joints(".*")

        self.potentials = torch.zeros(self.num_envs, dtype=torch.float32, device=self.sim.device)
        self.prev_potentials = torch.zeros_like(self.potentials)
        self.targets = torch.tensor([1000, 0, 0], dtype=torch.float32, device=self.sim.device).repeat(
            (self.num_envs, 1)
        )
        self.targets += self.scene.env_origins
        self.start_rotation = torch.tensor([1, 0, 0, 0], device=self.sim.device, dtype=torch.float32)
        self.up_vec = torch.tensor([0, 0, 1], dtype=torch.float32, device=self.sim.device).repeat((self.num_envs, 1))
        self.heading_vec = torch.tensor([1, 0, 0], dtype=torch.float32, device=self.sim.device).repeat(
            (self.num_envs, 1)
        )
        self.inv_start_rot = quat_conjugate(self.start_rotation).repeat((self.num_envs, 1))
        self.basis_vec0 = self.heading_vec.clone()
        self.basis_vec1 = self.up_vec.clone()
\end{lstlisting}
En primer lugar, se utiliza el constructor de \clase{DirectRLEnv}. Este permite instanciar todos aquellos valores que se necesitan por defecto, como la configuración o el número de entornos. Después se definen distintos atributos relevantes al procesado de las acciones. Se define la escala de las acciones, se transforma el vector de la ponderación de la fuerza, se declara un tensor completo a uno con la dimensión del vector de la ponderación y se recoge el nombre de las distintas articulaciones. Cabe resaltar en estos atributos el uso del método \metodo{torch.tensor} \cite{pytorch_docs}, que permite crear un tensor; y del método \metodo{torch.ones\_like} \cite{pytorch_docs}, que permite crear un tensor inicializado entero a uno con la dimensión del referenciado. Seguidamente, se definen todos los atributos relevantes al cálculo de las observaciones y las recompensas. Algunos como \atributo{potentials} o \atributo{prev\_potentials}, al tratarse más adelante, se inicializan a cero, mediante el método \metodo{torch.zeros} \cite{pytorch_docs}, en el cual se indica directamente la dimensión; o el método \metodo{torch.zeros\_like}, donde la dimensión se da indirectamente a través de un sensor. Otros, como \atributo{start\_rotation}, que indica la rotación inicial de la araña, o \atributo{heading\_vec}, que indica un vector de referencia para el avance, se definen directamente con \atributo{torch.tensor}. En algunos de ellos, se utiliza el método \atributo{torch.Tensor.repeat()} \cite{pytorch_docs}, que permite duplicar el tensor. En el caso de \atributo{inv\_start\_rot}, dónde se calcula el inverso de la rotación inicial, se repite dicho valor por el número de entornos en el eje 0 y por 1 en el eje 1, quedando un tensor de forma [num\_envs, 4].

El siguiente método que se define es \metodo{\_set\_up\_scene}. Este era el encargado de, con los elementos definidos en la configuración, crear la escena. El código de la definición de este método se muestra a continuación (código \ref{lst:suploc}).
\begin{lstlisting}[style=mypython, caption={Definición del método \metodo{\_set\_up\_scene(self)} de la clase \clase{LocomotionEnv}},  label={lst:suploc}]
    def _setup_scene(self):
        self.robot = Articulation(self.cfg.robot)
        # inclusion del plano del entorno
        self.cfg.terrain.num_envs = self.scene.cfg.num_envs
        self.cfg.terrain.env_spacing = self.scene.cfg.env_spacing
        self.terrain = self.cfg.terrain.class_type(self.cfg.terrain)
        # clonar y replicar
        self.scene.clone_environments(copy_from_source=False)
        # incluir la articulacion a la escena
        self.scene.articulations["robot"] = self.robot
        # add lights
        light_cfg = sim_utils.DomeLightCfg(intensity=2000.0, color=(0.75, 0.75, 0.75))
        light_cfg.func("/World/Light", light_cfg)
\end{lstlisting}
Lo primero que realiza este método es instanciar una clase \clase{Articulation} mediante la configuración del robot, almacena en la configuración del entorno. Después, se completa la configuración del terreno con el número de entornos y el espacio entre ellos definido, y se instancia de la misma manera. Seguidamente, se define la forma de clonado de los entornos, en este caso, al negar \atributo{copy\_from\_source}, los entornos clonados no heredan los estados del original, siendo así independientes de este. El robot, por otro lado, al ser la pieza central, tiene un hueco asignado dentro de la escena, por lo que se debe asociar a esta, a pesar de tenerlo declarado también en otra variable. Por último, se configuran las luces para la visualización de la escena.

Dos métodos, que en este caso van estrechamente relacionados, son los métodos \metodo{\_pre\_physics\_step()} y \metodo{\_apply\_action()}. Pese a que tienen objetivos distintos (como se vio en el apartado \ref{ap:DirectRLEnv}), en este caso ambos métodos tratan las acciones; esto se puede observar en el siguiente código (código \ref{lst:pfaaloc})
\begin{lstlisting}[style=mypython, caption={Definición del método \metodo{\_pre\_physics\_step(self)} y \metodo{\_apply\_action} de la clase \clase{LocomotionEnv}},  label={lst:pfaaloc}]
    def _pre_physics_step(self, actions: torch.Tensor):
        self.actions = actions.clone()

    def _apply_action(self):
        forces = self.action_scale * self.joint_gears * self.actions
        self.robot.set_joint_effort_target(forces, joint_ids=self._joint_dof_idx)
\end{lstlisting}
En el método \metodo{\_pre\_physics\_step} se guarda la acción a realizar, indicada dentro del método. Por otro lado, en el método \metodo{\_apply\_action()}, se calculan las fuerzas, mediante los distintos parámetros de escala y proporción, y luego se aplican a las articulaciones del robot.

El siguiente método definido es \metodo{\_compute\_intermediate\_values(self)}. Este  se declara directamente en esta clase, no es heredado de su clase base. Su objetivo es realizar una serie de cálculos para determinar una serie de valores. Estos valores luego serán utilizados en distintas evaluaciones del proceso de aprendizaje. Este método contiene una función \metodo{compute\_intermediate\_rewards(...)}, definida fuera de la clase; esta es la que contiene estos cálculos. Esta definición se muestra en el siguiente código (código \ref{lst:civloc}).
\begin{lstlisting}[style=mypython, caption={Definición del método \metodo{compute\_intermediate\_rewards(...)}},  label={lst:civloc}]
    def compute_intermediate_values(
        targets: torch.Tensor,
        torso_position: torch.Tensor,
        torso_rotation: torch.Tensor,
        velocity: torch.Tensor,
        ang_velocity: torch.Tensor,
        dof_pos: torch.Tensor,
        dof_lower_limits: torch.Tensor,
        dof_upper_limits: torch.Tensor,
        inv_start_rot: torch.Tensor,
        basis_vec0: torch.Tensor,
        basis_vec1: torch.Tensor,
        potentials: torch.Tensor,
        prev_potentials: torch.Tensor,
        dt: float,
    ):
        to_target = targets - torso_position
        to_target[:, 2] = 0.0
    
        torso_quat, up_proj, heading_proj, up_vec, heading_vec = compute_heading_and_up(
            torso_rotation, inv_start_rot, to_target, basis_vec0, basis_vec1, 2
        )
    
        vel_loc, angvel_loc, roll, pitch, yaw, angle_to_target = compute_rot(
            torso_quat, velocity, ang_velocity, targets, torso_position
        )
    
        dof_pos_scaled = torch_utils.maths.unscale(dof_pos, dof_lower_limits, dof_upper_limits)
    
        to_target = targets - torso_position
        to_target[:, 2] = 0.0
        prev_potentials[:] = potentials
        potentials = -torch.norm(to_target, p=2, dim=-1) / dt
    
        return (
            up_proj,
            heading_proj,
            up_vec,
            heading_vec,
            vel_loc,
            angvel_loc,
            roll,
            pitch,
            yaw,
            angle_to_target,
            dof_pos_scaled,
            prev_potentials,
            potentials,
        )
\end{lstlisting}
A continuación, se van a analizar cada uno de los parámetros calculados:
\begin{itemize}
    \item \atributo{up\_proj}: proyección escalar del vector que indica la dirección superior del torso de la araña. Es decir, el vector normal al torso. Se calcula en referencia a la rotación del torso, su rotación inicial y la referencia antes dada de la dirección superior objetivo, [0, 0, 1].
    \item \atributo{heading\_proj}: Este atributo, de una misma manera que el anterior, indica el vector dirección de la araña, tomando de referencia el vector de dirección definido [1, 0, 0].
    \item \atributo{up\_vec}: el vector de la dirección superior. La proyección de este sobre el vector de dirección superior, [0, 0, 1], resultaría en el atributo \atributo{up\_proj}
    \item \atributo{heading\_vec}: de una misma manera que el anterior, este atributo representa el vector de la dirección. La proyección de este sobre el vector de dirección superior, [0, 0, 1], resulta en el atributo \atributo{heading\_proj}.
    \item \atributo{vel\_loc}: indica la velocidad con la cual se acerca al objetivo establecido en \atributo{targets}.
    \item \atributo{roll}, \atributo{pitch}, \atributo{yaw}: indica la rotación del torso de la araña en esa convención. Lo hace a partir del cuaterno almacenado en \atributo{torso\_quat}.
    \item \atributo{angle\_to\_target}: indica el angulo entre el vector de dirección y el vector hacia el objetivo.
    \item \atributo{dof\_pos\_scaled}: almacena la posición de las articulaciones desnormalizadas de sus límites.
    \item \atributo{prev\_potentials}: almacena el valor anterior de los potenciales.
    \item \atributo{potentials}: calcula si el robot se acerca o se aleja del objetivo.
\end{itemize}
Cada uno de estos elementos será relevante en la valoración y observación de los entornos, por lo que se deberá ejecutar en cada paso de la simulación.

A continuación, se define el método \metodo{\_get\_observations(self)}. Este, era un método de abstracto de la clase base y debe utilizarse para recoger las observaciones del entorno. Para ello, se conforma un tensor de un único eje donde vienen todos los valores de las observaciones. La dimensión de este único eje, será el número de entradas que tendrá la red neuronal. Este método, se define de la siguiente manera (código \ref{lst:obsloc}):
\begin{lstlisting}[style=mypython, caption={Definición del método \metodo{\_get\_observations(self)}},  label={lst:obsloc}]
    def _get_observations(self) -> dict:
    obs = torch.cat(
        (
            self.torso_position[:, 2].view(-1, 1),
            self.vel_loc,
            self.angvel_loc * self.cfg.angular_velocity_scale,
            normalize_angle(self.yaw).unsqueeze(-1),
            normalize_angle(self.roll).unsqueeze(-1),
            normalize_angle(self.angle_to_target).unsqueeze(-1),
            self.up_proj.unsqueeze(-1),
            self.heading_proj.unsqueeze(-1),
            self.dof_pos_scaled,
            self.dof_vel * self.cfg.dof_vel_scale,
            self.actions,
        ),
        dim=-1,
    )
    observations = {"policy": obs}
    return observations
\end{lstlisting}
Esta clase utiliza el método \metodo{torch.cat} para concatenar una serie de vectores. Estos vectores conformaran las distintas observaciones que se realizarán sobre el entorno. Entre ellas encontramos las siguientes:
\begin{itemize}
    \item \atributo{torso\_position[:, 2]}: La altura del torso.
    \item \atributo{vel\_loc}: La velocidad de aproximación al objetivo, calculada en \metodo{\_compute\_intermi\allowbreak diate\_rewards(self)}.
    \item \atributo{angvel\_loc}: La velocidad angular con la que se aproxima al objetivo, calculada en \metodo{\_compute\_intermidiate\_rewards(self)}.
    \item \atributo{yaw}, \atributo{row}: Ángulos de rotación sobre el eje X y el eje Z. Se normalizan ambos.
    \item \atributo{angle\_to\_target}: Ángulo respecto al objetivo, calculada en \metodo{\_compute\_intermi\allowbreak diate\_rewards(self)}.
    \item \atributo{up\_proj}: Proyección del normal al torso, calculada en \metodo{\_compute\_intermidiate\_ \allowbreak rewards(self)}.
    \item \atributo{heading\_prol}: Proyección del vector de dirección del torso, calculada en \metodo{\_compute\_\allowbreak intermidiate\_rewards(self)}.
    \item \atributo{dof\_pos\_scaled}: La posición de las articulaciones escalada, calculada en \metodo{\_compute\_\allowbreak intermidiate\_rewards(self)}.
    \item \atributo{dof\_vel}: La velocidad de las articulaciones, la cual se escala multiplicando por el parámetro correspondiente.
    \item \atributo{actions}: La última acción registrada.
\end{itemize}
Cabe resaltar que todas las variables deben darse con dos ejes, uno de dimensión igual al número de entornos y otro con dimensión igual al tamaño de dicha observación. Para ello, en algunos casos se usa la función \metodo{torch.unsqueeze(dim)} \cite{torch_api}. Está funciona añade un eje en la dimensión indicada. Por ejemplo, la variable \atributo{yaw}, es del tipo [num\_envs]; con esta función se convierte en [num\_envs, 1]. También se usa el método \metodo(torch.Tensor.view(*shape)) \cite{torch_api}, que cambia la forma, indicando el tamaño. El valor -1 calcula automáticamente la dimensión del eje. En este caso, se obtendría un vector de la forma [num\_envs, 1]. Completando el método \metodo{torch.cat}, se indica el eje sobre el que se hace la concatenación; al indicar -1, sería el eje de fondo. La observación luego se almacena en un diccionario con la cadena "policy" y se devuelve.

Una vez obtenidas las observaciones, se continua utilizando los parámetros de nuestro entorno para calcular las recompensas. Esto se hace con el método heredado \metodo{\_get\_rewards(self)}. Este método, igual que para \metodo{\_compute\_intermidiate\_values\allowbreak (self)}, utiliza una función definida fuera de la clase: \metodo{compute\_rewards(...)}. El retorno de este método será el retorno de la clase principal, por lo que se analizará esta más detenidamente. A continuación, se muestra el código (código \ref{lst:rewloc}):
\begin{lstlisting}[style=mypython, caption={Definición del método \metodo{compute\_rewards(self)}},  label={lst:rewloc}]
def compute_rewards(
    actions: torch.Tensor,
    reset_terminated: torch.Tensor,
    up_weight: float,
    heading_weight: float,
    heading_proj: torch.Tensor,
    up_proj: torch.Tensor,
    dof_vel: torch.Tensor,
    dof_pos_scaled: torch.Tensor,
    potentials: torch.Tensor,
    prev_potentials: torch.Tensor,
    actions_cost_scale: float,
    energy_cost_scale: float,
    dof_vel_scale: float,
    death_cost: float,
    alive_reward_scale: float,
    motor_effort_ratio: torch.Tensor,
):
    heading_weight_tensor = torch.ones_like(heading_proj) * heading_weight
    heading_reward = torch.where(heading_proj > 0.8, heading_weight_tensor, heading_weight * heading_proj / 0.8)

    # aligning up axis of robot and environment
    up_reward = torch.zeros_like(heading_reward)
    up_reward = torch.where(up_proj > 0.93, up_reward + up_weight, up_reward)

    # energy penalty for movement
    actions_cost = torch.sum(actions**2, dim=-1)
    electricity_cost = torch.sum(
        torch.abs(actions * dof_vel * dof_vel_scale) * motor_effort_ratio.unsqueeze(0),
        dim=-1,
    )

    # dof at limit cost
    dof_at_limit_cost = torch.sum(dof_pos_scaled > 0.98, dim=-1)

    # reward for duration of staying alive
    alive_reward = torch.ones_like(potentials) * alive_reward_scale
    progress_reward = potentials - prev_potentials

    total_reward = (
        progress_reward
        + alive_reward
        + up_reward
        + heading_reward
        - actions_cost_scale * actions_cost
        - energy_cost_scale * electricity_cost
        - dof_at_limit_cost
    )
    # adjust reward for fallen agents
    total_reward = torch.where(reset_terminated, torch.ones_like(total_reward) * death_cost, total_reward)
    return total_reward
\end{lstlisting}
En este método, se calculan una serie de recompensas que luego se suman al final para obtener un único valor numérico. El valor final, almacenado en \atributo{total\_reward}, puede tener dos valores: un valor fijo derivado del coste de terminación o un valor variable dependiendo de las recompensas obtenidas. La selección de este valor se da mediante el método \metodo{torch.where} \cite{pytorch_docs}. En él, se introduce una tensor de valores booleanos que indican cuales de los entornos han finalizado. En aquellos que el entrono haya finalizado, se guardara el coste de terminación como recompensa, y en el resto, la recompensa variable. Esta recompensa variable esta conformada por las siguientes recompensas:
\begin{itemize}
    \item \atributo{progress\_reward}: recompensa por tender a acercarse o tender a alejarse. Se calcula a partir de los potenciales, siendo esta recompensa la variación de este.
    \item \atributo{alive\_reward}: recompensa por continuar el ejercicio.
    \item \atributo{up\_reward}: recompensa por mantener el torso paralelo al suelo. 
    \item \atributo{heading\_reward}: recompensa por mantener la dirección del torso hacia el objetivo.
    \item \atributo{action\_cost}: penalización por el uso de acciones.
    \item \atributo{electricity\_cost}: penalización por el gasto energético.
    \item \atributo{dof\_at\_limit\_cost}: penalización por forzar las articulaciones a si límite.
\end{itemize}
Todas las recompensas son tensores con forma [num\_envs, 1]. Estas recompensas se pueden calcular de maneras distintas. Por ejemplo, \atributo{up\_reward} se calcula de manera absoluta. Si sobrepasa la proyección un valor fijo, obtiene la totalidad de la recompensa, de lo contrario, obtiene un valor nulo. \atributo{heading\_reward}, por otro lado, se calcula de manera que se toma un valor estándar, y se obtiene una recompensa progresiva hasta llegar a este, donde se obtiene la recompensa total. Estas distintas formas de calcular las recompensas dependerán de lo que se quiera conseguir. En este caso, se es más flexible con la dirección, pero se busca mantener el torso paralelo al suelo dentro del rango indicado.

Dentro de estas recompensas toma relevancia el atributo  \atributo{reset\_\allowbreak terminated}, el cual indica los entornos que han finalizado. Este atributo se calcula con el método \metodo{\_get\_\allowbreak dones(self)}, que resulta ser el siguiente método definido. Este método también es heredado de la clase base. Se define mediante el siguiente código (código \ref{lst:doneloc}):
\begin{lstlisting}[style=mypython, caption={Definición del método \metodo{\_get\_dones(self)}},  label={lst:doneloc}]
    def _get_dones(self) -> tuple[torch.Tensor, torch.Tensor]:
        self._compute_intermediate_values()
        time_out = self.episode_length_buf >= self.max_episode_length - 1
        died = self.torso_position[:, 2] < self.cfg.termination_height
        return died, time_out
\end{lstlisting}
En esta clase, antes de calcular las terminaciones, se llama al método \metodo{\_compute\_\allowbreak intermi\allowbreak diate\_rewards(self)}. Esto se debe a que el método \metodo{\_get\_dones(self)}, se declara al comienzo de cada paso de simulación, por lo que de este modo, se calculan en cada momento los valores necesarios. Después de esto, se comprueban lo dos casos de finalización. El primero de ellos se da cuando se supera el tiempo máximo de episodio; el segundo cuando la posición del torso baja de un límite. En ambos casos, se almacena un tensor de forma [num\_envs, 1], donde el eje de fondo guarda un valor booleano. Los dos tensores tienen funciones distintas. Ambos indicarán una terminación del episodio del entorno, pero solo los indicados en \atributo{died} tendrán penalización \cite[isaaclab.envs, DirectRLEnv, step(self, action)]{isaaclab_api}.

Estos tensores indican por lo tanto que entornos deben reiniciarse, pues han llegado al final de su episodio. Este reinicio se define dentro del método \metodo{\_reset\_idx(self)}, también heredado de la clase base. El código del método es el siguiente:
\begin{lstlisting}[style=mypython, caption={Definición del método \metodo{\_reset\_idx(self)}},  label={lst:doneloc}]
    def _reset_idx(self, env_ids: torch.Tensor | None):
        if env_ids is None or len(env_ids) == self.num_envs:
            env_ids = self.robot._ALL_INDICES
        self.robot.reset(env_ids)
        super()._reset_idx(env_ids)

        joint_pos = self.robot.data.default_joint_pos[env_ids]
        joint_vel = self.robot.data.default_joint_vel[env_ids]
        default_root_state = self.robot.data.default_root_state[env_ids]
        default_root_state[:, :3] += self.scene.env_origins[env_ids]

        self.robot.write_root_pose_to_sim(default_root_state[:, :7], env_ids)
        self.robot.write_root_velocity_to_sim(default_root_state[:, 7:], env_ids)
        self.robot.write_joint_state_to_sim(joint_pos, joint_vel, None, env_ids)

        to_target = self.targets[env_ids] - default_root_state[:, :3]
        to_target[:, 2] = 0.0
        self.potentials[env_ids] = -torch.norm(to_target, p=2, dim=-1) / self.cfg.sim.dt

        self._compute_intermediate_values()
\end{lstlisting}
Este método recibe un tensor con los entornos a reiniciar indicados en \atributo{env\_ids}. En este entorno, al tener como único elemento el robot, primero se identifica cada uno de los robots a reiniciar. Después, se hace el reinicio generalizado, mediante el método de la clase padre. Una vez reiniciado los entornos, se lleva el robot al estado y posición original, usando las variables almacenadas dentro de la clase. También se vuelve a calcular el objetivo, ahora desde la posición original; así como los potenciales. Por último, al haber variado las posiciones y estados del robot, se deben volver a calcular los valores intermedios.

En esta clase, se definen todos los aspectos relevantes del entorno. De esta manera, se define los pasos a seguir dentro de las simulación, indicando cada una de las interacciones entre el entorno, la simulación y el agente. Más adelante, en el capítulo siguiente, se podrá apreciar la diferencia con la arquitectura por manejadores. En esta clase, se ha podido estudiar profundamente como se estructuran y calculan las observaciones y las recompensas, así como la realización de reinicios y la creación de los elementos de la escena; en el modo por manejadores, el enfoque estará situado en las clases y no tanto en la definición de los métodos.

Antes de pasar a estudiar como se registra este entorno queda por analizar la clase final, \clase{AntEnv}, que heredará sobre esta clase y modificará los distintos atributos de esta para adaptarla a su caso específico. Esta clase no se podría instanciar todavía, pues faltaría por definir atributos como los pesos de las recompensas o el robot del entorno. Todos estos atributos ya han sido definidos dentro de la configuración de la clase. La clase \clase{AntEnv} queda definida en el mismo archivo que la clase \clase{AntEnvCfg}. Se muestra su definición en el código \ref{lst:antenv}. Esta solo indica el tipo de configuración que recibe, \clase{AntEnvCfg}, y utiliza el constructor de la clase padre para instanciar el entorno. Al tener todos los parámetros definidos dentro de la configuración y heredar de la clase \clase{LocomotionEnv}, obteniendo la definición de las interacciones y construcciones; no hace falta definir ningún elemento más, pudiendo pasar directamente a su registro.
\begin{lstlisting}[style=mypython, caption={Definición de la clase \clase{AntEnv}},  label={lst:antenv}]
class AntEnv(LocomotionEnv):
    cfg: AntEnvCfg

    def __init__(self, cfg: AntEnvCfg, render_mode: str | None = None, **kwargs):
        super().__init__(cfg, render_mode, **kwargs)
\end{lstlisting}

\section{Registro del Entorno}
\label{ap:regisarana}
El registro del entorno sirve para señalizar una tarea para su entrenamiento. Este proceso se hace en dos pasos: configurar el algoritmo de entrenamiento y señalizarlo dentro de \emph{gymnasium}.

La primera parte de este proceso se realiza en el archivo \verb|source/isaaclab_tasks| \verb|/isaaclab_tasks/direct/ant/agents/rsl_rl_ppo_cfg.py|. En este archivo se definen los principales parámetros del entrenamiento. Primero, se definen parámetros para la propia ejecución de la simulación junto con el entrenamiento. Entre ellos, el número máximo de iteraciones, el guardado de la política o el nombre del ejercicio. Después se define el formato de la política, indicando el ruido y los parámetros propios del actor y el crítico que formaran esta. Por último, se define el formato del algoritmo, indicando parámetros propios del PPO a implementar. Como el enfoque de este trabajo se centra en los entornos, no se entrará en gran detalle sobre esta configuración.

La segunda parte de este proceso registra el entorno y la configuración del algoritmo dentro de \emph{gymnasium} \cite{towers2024gymnasium}. Para ello se define el siguiente código (código \ref{lst:antgym}):
\begin{lstlisting}[style=mypython, caption={Registro del entorno \emph{Ant}},  label={lst:antgym}]
gym.register(
    id="Isaac-Ant-Direct-v0",
    entry_point=f"{__name__}.ant_env:AntEnv",
    disable_env_checker=True,
    kwargs={
        "env_cfg_entry_point": f"{__name__}.ant_env:AntEnvCfg",
        "rl_games_cfg_entry_point": f"{agents.__name__}:rl_games_ppo_cfg.yaml",
        "rsl_rl_cfg_entry_point": f"{agents.__name__}.rsl_rl_ppo_cfg:AntPPORunnerCfg",
        "skrl_cfg_entry_point": f"{agents.__name__}:skrl_ppo_cfg.yaml",
    },
)
\end{lstlisting}
Se utiliza el método \metodo{gymnasium.register}, el cual registra el entorno junto a la configuración del agente. En el entrenamiento, se extraen ambas partes y se implementan para realizar el entrenamiento. Este proceso se realiza a través de los scrips propios de la librería de aprendizaje y es el mismo para cualquier ejercicio. Por tanto, el ejercicio de entrenamiento dependerá de dos factores. Por un lado, el entorno; este se define mediante el atributo \atributo{entry\_point}, que señaliza la clase del entorno principal, y, dentro del atributo \atributo{kwargs}, la cadena asociada a \atributo{env\_cfg\_entry\_point}, que señaliza la clase configuración del entorno. Dentro de \atributo{kwargs} se define también la otra parte del ejercicio, la configuración del agente. En el párrafo anterior, se ha centrado en la biblioteca \emph{rsl\_rl}, la que se usará en este proyecto; sin embargo, en este ejercicio se definen también el resto de bibliotecas. Como se puede observar, se puede llamar a un archivo o, en este caso, a una clase \clase{AntPPORunnerCfg}.

Con la clase registrada, se pueden realizar tanto los ejercicios de entrenamiento como de evaluación. En el siguiente apartado, veremos como ejecutar ambos procesos a través de la terminal.

\section{Aprendizaje y Evaluación}
\label{ap:entrearana}
Una vez estudiado tanto el registro del entorno como su construcción, se va a proceder a realizar el ejercicio de entrenamiento. Para ello, se utilizarán los archivos contenidos en \verb|IsaacLab/scripts/reinforcement_learning/rsl_rl|. Dentro de esta carpeta se encuentran los código para realizar el aprendizaje, en \verb|train.py|, y la evaluación, \verb|play.py|. Ambos códigos son propios de la biblioteca \emph{rsl\_rl} y la herramienta IsaacLab, por lo que no se van a analizar individualmente. En este apartado se va a estudiar como ejecutar ambos códigos a través de la terminal, indicando con las etiquetas todos los datos necesarios.

El primer paso será realizar un entrenamiento de prueba, para ver que el entorno se genere adecuadamente y el entrenamiento se pueda realizar de la forma indicada. Para ello, utilizaremos la siguiente sentencia por terminal (código \ref{lst:entantvis}):
\begin{lstlisting}[language=bash, label={lst:entantvis}, caption={Entrenamiento de prueba para la locomoción de \emph{Ant}}]
    python scripts/reinforcement_learning/rsl_rl/train.py --task
    Isaac-Ant-Direct-v0 --num_envs 4
\end{lstlisting}
Con esta sentencia ejecutamos el entrenamiento, indicando la tarea que se quiere entrenar y el número de entornos a utilizar. En un primer momento se van a utilizar un número pequeño de entornos para valorar la construcción del entorno dentro del simulador. Al ejecutarlo, se abre la aplicación IsaacSim y se generan cuando de los robots a entrenar (figura \ref{fig:entvis}). Se observa como los cuatro robots mueven sus articulaciones de forma errática y al caer su torso sobre el suelo se reinicia el entorno.

\begin{figure}[ht]
    \label{fig:entvis}
    \centering
    \includegraphics[width=\linewidth]{imagenes/antentvis.png}
    \caption{Prueba para el entrenamiento del robot araña}
\end{figure}

Una vez se ha realizado una prueba satisfactoria, se puede proceder a realizar el entrenamiento completo. Para ello, se deberán generar una mayor cantidad de entornos, con el objetivo de acelerar el proceso. Para ello, se utilizará la siguiente sentencia (código \ref{entanthls}):
\begin{lstlisting}[language=bash, label={lst:entantvis}, caption={Entrenamiento de prueba para la locomoción de \emph{Ant}}]
    python scripts/reinforcement_learning/rsl_rl/train.py --task
    Isaac-Ant-Direct-v0 --num_envs 512 --headless
\end{lstlisting}
Con esta sentencia, ejecutamos el mismo código de entrenamiento, pero aumentando el número de entornos a 512 y utilizando la nueva etiqueta \verb|--headless|. Esta última etiqueta permite ejecutar el entrenamiento sin visualizarlo en el simulador, lo que permite acortar el tiempo de entrenamiento al usar menos recursos. Para poder evaluar el proceso de entrenamiento se obtiene una serie de información en la terminal; esta información se muestra en la figura \ref{fig:entanthls1}. Dentro del cajetín se obtiene información acerca del número de iteraciones o la longitud del episodio. El valor más relevante para evaluar el proceso de entrenamiento es la recompensa media. Está no solo ayuda al robot al aprendizaje si no da información sobre como se desenvuelve el agente dentro en la tarea. En su última iteración (figura \ref{fig:antenthls999}), la recompensa media asciende a 12055.38, por lo que se puede intuir que la araña realiza correctamente el ejercicio de locomoción. 

\begin{figure}[ht]
    \label{fig:antenthls1}
    \centering
    \includegraphics[width=\linewidth]{imagenes/antenthls1.png}
    \caption{Primeras iteraciones del ejercicio de aprendizaje}
\end{figure}

\begin{figure}[ht]
    \label{fig:antenthls999}
    \centering
    \includegraphics[width=\linewidth]{imagenes/antenthls999.png}
    \caption{Última iteración del ejercicio de aprendizaje}
\end{figure}

El resultado de este aprendizaje es una serie de modelos, almacenados en la carpeta \verb|IsaacLab/logs/rsl_rl|. Dentro de esta carpeta se almacena un modelo cada 50 iteraciones (este parámetro se puede variar dentro de la configuración del agente). Estos modelos contienen la red neuronal con la política implementada en cada iteración. Sin embargo, estos modelos no se pueden utilizar, para ello, primero se debe seleccionar uno de los modelos y evaluarlo con \verb|play.py|. Para ello, se utiliza la siguiente sentencia (código \ref{lst:antenteva}):
\FloatBarrier
\begin{lstlisting}[language=bash, label={lst:antenteva}, caption={Entrenamiento de prueba para la locomoción de \emph{Ant}}]
    python scripts/reinforcement_learning/rsl_rl/play.py --task
    Isaac-Ant-Direct-v0 --num_envs 4 --checkpoint logs/rsl_rl/
    ant_direct/2025-12-29_10-04-20/model_900.pt
\end{lstlisting}
\FloatBarrier
Al ejecutar esta sentencia se abre el simulador, pudiendo estudiar el movimiento de la araña. A través de una inspección visual se puede confirmar que el movimiento de la araña es correcto, por lo que el valor alto de recompensa se adecua a las expectativas. Añadido a la evaluación, al ejecutar este fichero se obtiene un archivo \verb|policy.pt|, almacenado en el directorio \verb|logs/rsl_rl/ant_direct/load_run/exported|, el cual si puede ser exportado y utilizado. Por tanto, esta política ya podría ser llevada al robot real, sin embargo, este ejercicio no está diseñado para dicha implementación. Para poder utilizar este entrenamiento se deben tener más factores en cuenta, lo que se estudiará en el siguiente apartado.

\section{Posibles mejoras}

Con vistas a los futuros proyectos dentro de la ETSIDI se van a proponer e integrar dos principales mejoras. Estas se introducirán en este trabajo, pero su expansión e implementación final se darán en futuros proyectos. La primera de ellas será cambiar el terreno donde se entrena la locomoción. En el ejercicio realizado, el terreno es completamente plano, algo poco usual en el mundo real. La segunda mejora a integrar es el uso de cámaras. Estas podrán ayudar a evitar obstáculos o mejorar la locomoción en terrenos irregulares.

\subsection{Terreno irregular}
Para introducir el terreno irregular se va utilizar una nueva clase de configuración \clase{TerrainGeneratorCfg}. Esa clase permite definir un nuevo tipo de terreno, que luego es seleccionado dentro de \clase{TerrainImporterCfg}. En un primer lugar, se define la generación del terreno; después, mediante los atributos \atributo{terran\_type} y \atributo{terrain\_generator}, se genera el terreno definido.

La definición de la generación del terreno es la siguiente (código \ref{lst:antterr}):
\begin{lstlisting}[style=mypython, caption={Definición del terreno con relieve a generar.},  label={lst:antterr}]
RANDOM_ROUGH_CFG = terrain_gen.TerrainGeneratorCfg(
    size = (100.0, 100.0),
    num_rows = 1,
    num_cols = 1,
    horizontal_scale=0.1,
    vertical_scale=0.005,
    slope_threshold=0.75,
    use_cache=False,
    sub_terrains = {
        "random_rough": terrain_gen.HfRandomUniformTerrainCfg(
            proportion= 1.0, noise_range=(0.02, 0.10), noise_step=0.002
        )
    }
)
\end{lstlisting}
Mediante este código, se define, de forma externa a la clase de configuración, una constante para la generación del terreno. Esta clase define un terreno irregular cuadrado de 100 metros de lado. Para generar el propio terreno, se indica como único sub-terreno una clase \clase{HfRandomUniformTerrainCfg} \cite{isaaclab_api}, característica de un terreno rugoso. Con su constructor, se indica la proporción de este terrero (al ser el único sub-terreno, 1), la altura máxima y mínima del terreno (indicado mediante \atributo{noise\_range}) y la definición de dicho terreno, es decir, la distancia mínima entre dos puntos adyacentes. En otros casos, se puede hacer uso del concepto de sub-terreno para crear distintas zonas con distintos tipos de terreno. En este caso, al ser meramente una prueba, basta con generar un único tipo de terreno.

\subsection{Cámaras}
La otra mejora a implementar es el uso de cámaras. Existen dos tipos de cámaras para incluir en la simulación, cámaras normales, generadas por \clase{CameraCfg}, y cámaras en mosaico \clase{TiledCameraCfg}. La cámara en mosaico se genera de una misma forma que la cámara normal. Su diferencia radica en el procesamiento interno por la GPU \cite{isaaclab_doc}, mejorando el procesamiento. Por esto, en este caso se usará directamente la cámara en mosaico. El código para su configuración es el siguiente, (código \ref{lst:camant}):
\begin{lstlisting}[style=mypython, caption={Definición de la configuraciíon de la cámara a generar.},  label={lst:camant}]
    camera: TiledCameraCfg = TiledCameraCfg(
    prim_path="/World/envs/env_.*/Robot/torso/FrontCamera",
    update_period= 0.1,
    height = 64,
    width= 64,
    data_types= ["rgb", "depth"],
    spawn= PinholeCameraCfg(
        focal_length = 24.0, focus_distance = 400, clipping_range=(0.1, 20)
    ),
    offset= TiledCameraCfg.OffsetCfg(pos = (0.3, 0, 0), rot = (0.9239,0,-0.3827,0), convention = "world")
    )
\end{lstlisting}
En este código, incluido dentro de la clase de configuración, \clase{RoughAntEnvCfg}, se define la configuración de la cámara. La cámara se acopla al lateral del torso de la araña, con una rotación sobre el eje y de -45 grados, de modo que apunte a la nueva superficie rugosa generada. Se permiten recopilar dos tipos de datos: los colores rgb y la profundidad. Se ajusta también los parámetros de la cámara, como el tipo de cámara a usar, mediante la clase \clase{PinholeCameraCfg}, el tiempo de actualización y la resolución. Esta configuración debe ser instanciada manualmente. Para ello, al definir la nueva clase principal para el entrenamiento, \clase{RoughAntEnv}, se debe re-definir la función \metodo{\_set\_up\_scene(self)} para generar la cámara (código \ref{lst:stuprou}).
\begin{lstlisting}[style=mypython, caption={Implementación de la cámara en la escena.},  label={lst:stuprou}]
    def _setup_scene(self):
        super()._setup_scene()
        self.camera = TiledCamera(self.cfg.camera)
\end{lstlisting}

El mayor inconveniente de utilizar las cámaras es el tamaño de la información. Para cualquier cámara, si se quiere usar la información de la profundidad, se deberá tratar con un tensor de forma [num\_envs, height, width, 1]. Esto quiere decir que para cada entorno, tomando el ejemplo propuesto, si se quisiese usar la imagen al completo como observación, se tendría un vector de observaciones con una dimensión superior de 4096 elementos; usando de ejemplo una resolución pobre como la propuesta. Esto hace su uso inviable en la mayoría de casos. Sin embargo, se pueden sortear estas dificultades realizando el procesamiento de imágenes externamente o pre-procesando la información para reducir el tamaño de la observación.

Este ejercicio se deja para futuros trabajos. Este trabajo pretende cimentar las bases para que otros alumnos o interesados puedan trabajar en este campo. El uso de visión artificial junto con la inteligencia artificial sobresale como un tema interesante para su estudio.

Todas estas mejoras se pueden encontrar en el proyecto preparado para el trabajo, incluido en los anexos. El archivo donde se implementan dichas mejoras es en el \verb|source/ARMetaToolPG/ARMetaToolPG/tasks/direct/ant/rough_ant.py|

Analizado este problema, se va a proceder a analizar un nuevo ejemplo. A diferencia de este, en él se utiliza el modo por manejadores.
\chapter{Reach}

En este capítulo se va estudiar un nuevo ejercicio de aprendizaje por refuerzo. Para este caso, se estudiará la construcción de entornos por manejadores. Debido a la constitución del código en esta forma de trabajar, el análisis será menos profundo. Esta forma permite estructurar el código de manera más superficial, organizándose en dos elementos: manejadores y términos; donde el entorno tiene un número de manejadores y estos tienen un número de términos. 

Para el estudió se desarrollará y analizará el diagrama de clases. Dentro de este diagrama de clases se analizará su pieza central, \clase{UR3EnvCfg}, la cual definirá la estructura de manejadores. Después se estudiarán, al igual que en el capítulo anterior, el registro, entrenamiento y evaluación del ejercicio de entrenamiento. Por último, se implementarán y propondrán algunas mejoras para este caso. Antes de todo esto, se va a presentar el caso de estudio.

\section{Descripción del caso práctico}
Para este segundo caso de estudió, se pasa al ámbito de la manipulación. Se va a realizar un ejercicio \emph{reach}, un paso previo a cualquier problema de manipulación. En él, se buscará llevar la última articulación de un robot a una posición y orientación concreta. El robot elegido es el UR3, de \emph{Universal Robots} \cite{robot_ur3e}. Se ha escogido este robot por dos motivos: la disponibilidad en el laboratorio de este robot y el uso de este robot en el proyecto MetaTool. El caso, por otro lado, ha sido escogido para introducir la forma de programación por manejadores y rotar hacia el enfoque de trabajo de MetaTool.

El código no es propio de este proyecto. En la mayoría de casos de aprendizaje no es necesario implementar desde cero el código. Gracias al gran volumen de ejemplos, lo común es partir de un ejemplo y realizar modificaciones al código. Esta forma de trabajar es la que se ha implementado para realizar este caso, partiendo del código de IsaacLab contenido en \verb|source/isaaclab_tasks/isaaclab_tasks/manager_based/manipulation/reach/reach_env_cfg.py| dentro de la herramienta IsaacLab. Dentro del análisis se indicarán los fragmentos modificados y se analizará el código al completo.

A continuación, se va dar comienzo al estudio del código, analizando el diagrama de clases desarrollado.

\section{Diagrama de clases}

En la figura \ref{UMLreach} se muestra el diagrama de clases preparado para este ejemplo. A primera vista, se puede observar que los elementos del diagram orbitan alrededor de la clase \clase{ReachEnvCfg}. Esta clase hereda de \clase{ManagerBasedCfg} y sirve de base para las del entorno específico \clase{UR3ReachEnvCfg} y \clase{UR3ReachEnvCfg\_PLAY}. De una misma manera que el ejemplo anterior, existe una clase para una tarea común, en este caso del \emph{reach} sería la clase \clase{ReachEnvCfg}.

La clase \clase{ReachEnvCfg} definiría la tarea general a implementar. Para cada aspecto general tendría un manejador distinto, los cuales se abalizarán detenidamente en futuros apartados. Como se viene comentando, cada uno de estos manejadores tiene como atributos distintos términos, que resumen cada una de las distintas partes de ese aspecto. Por ejemplo, el manejador \clase{RewardsCfg} esta constituido por una serie de términos \clase{RewTerm}.

A partir de esta clase heredan \clase{UR3ReachEnvCfg} y \clase{UR3ReachEnvCfg\_PLAY}; la primera de estas pensada para el entrenamiento y la segunda para la manipulación. Cabe notar que no se hereda a partir de la clase principal \clase{ManagerBasedRLEnv}, sino únicamente de las configuraciones. Esto ocurre ya que, a diferencia de la manera directa, en esta forma de programar se definen únicamente las piezas y elementos de las interacciones, los procesos propios de esta vienen ya definidos en la clase principal base.

En el siguiente apartado, se van a analizar cada uno de los manejadores individualmente, analizando el código utilizado en cada uno de ellos.

\begin{landscape}
\begin{figure}[ht]
    \label{UMLreach}
    \centering
    \includegraphics[width=\linewidth]{imagenes/ManagerBasedUML.pdf}
    \caption{Diagrama UML del ejemplo \emph{reach}, programación por manejadores.}
\end{figure}
\end{landscape}

\section{Manejadores}

En este apartado se van a estudiar cada uno de los manejadores detenidamente, estudiando su función y definición en su código. Todos los manejadores y clases se definen en el archivo \ver|source/isaaclab_tasks/isaaclab_tasks/manager_based/manipulation/reach/reach_env_cfg.py|

\paragraph{ReachSceneCfg}
La primera clase definida dentro del código no es un manejador como tal, sino la clase con la que se definen los elementos de la escena. No obstante, pese a no encontrarse dentro de los manejadores, se trata dentro del código como uno, definiendo los distintos elementos que lo componen y dejando su creación final a la clase base. Al estar tratando con la clase de configuración de la tarea base, no se define ningún elemento específico. Todos ellos se definen a partir de la clase \clase{AssetBaseCfg}, la clase básica para la definición de primitivos \cite{isaaclab_api}. 

El código utilizado para la definición de esta escena es el siguiente (código \ref{lst:scnrea}):
\begin{lstlisting}[style=mypython, caption={Definición de la escena mediante la clase \clase{ReachSceneCfg}},  label={lst:scnrea}]
@configclass
class ReachSceneCfg(InteractiveSceneCfg):
    """Configuration for the scene with a robotic arm."""

    # world
    ground = AssetBaseCfg(
        prim_path="/World/ground",
        spawn=sim_utils.GroundPlaneCfg(),
        init_state=AssetBaseCfg.InitialStateCfg(pos=(0.0, 0.0, -1.05)),
    )

    table = AssetBaseCfg(
        prim_path="{ENV_REGEX_NS}/Table",
        spawn=sim_utils.UsdFileCfg(
            usd_path=f"{ISAAC_NUCLEUS_DIR}/Props/Mounts/SeattleLabTable/table_instanceable.usd",
        ),
        init_state=AssetBaseCfg.InitialStateCfg(pos=(0.55, 0.0, 0.0), rot=(0.70711, 0.0, 0.0, 0.70711)),
    )

    # robots
    robot: ArticulationCfg = MISSING

    # lights
    light = AssetBaseCfg(
        prim_path="/World/light",
        spawn=sim_utils.DomeLightCfg(color=(0.75, 0.75, 0.75), intensity=2500.0),
    )
\end{lstlisting}
Se declaran 4 elementos a crear en la escena. El primero de ellos es el plano general, almacenado en la variable \atributo{ground}. Este plano se define utilizando el plano por defecto, situándolo a aproximadamente un metro del origen. Después, se define la mesa a utilizar. Para ello, se utiliza una mesa propia de IsaacLab y se sitúa (referido a su centro de coordenadas) aproximadamente medio metro del origen de la escena. Seguidamente, se declara el robot. Al ser este un elemento específico de la aplicación se utiliza la constante \atributo{MISSING}, para poder reemplazarlo más adelante. Por último, se define el tipo de iluminación. Cada uno de estos elementos conforman un primitivo, los cuales se almacenan dentro de cada entorno (con la constante \atributo{ENV\_REGEX\_NS}) o en el mundo (con el directorio \verb|/World/|). De esta manera queda estructurada la escena con sus elementos para que la clase \clase{ManagerBasedRLEnv} pueda crearla directamente.

\paragraph{CommandsCfg}
La siguiente clase definida trata, esta vez sí, del primer manejador, \clase{CommandsCfg}. Este manejador se encarga de generar objetivos para el ejercicio de entrenamiento. En el caso en cuestión, el objetivo es poder llevar la última articulación a una posición y orientación concreta. Para poder llevarla a distintos puntos, se debe entrenar para un rango de puntos; si se entrenase para un punto concreto, solo podría llegar a este. 

Con esta clase se pretende generar una serie de puntos objetivo para entrenar el movimiento hacia estos. El código sería el siguiente (código \ref{lst:cmdrea}):
\begin{lstlisting}[style=mypython, caption={Definición de los objetivos de entrenamiento mediante la clase \clase{CommandsCfg}},  label={lst:cmdrea}]
@configclass
class CommandsCfg:
    """Command terms for the MDP."""

    ee_pose = mdp.UniformPoseCommandCfg(
        asset_name="robot",
        body_name=MISSING,
        resampling_time_range=(4.0, 4.0),
        debug_vis=True,
        ranges=mdp.UniformPoseCommandCfg.Ranges(
            pos_x=(0.35, 0.65),
            pos_y=(-0.2, 0.2),
            pos_z=(0.15, 0.5),
            roll=(0.0, 0.0),
            pitch=MISSING,  # depends on end-effector axis
            yaw=(-3.14, 3.14),
        ),
    )
\end{lstlisting}
Dentro de la clase, se define un única variable donde se almacenarán las posiciones objetivo. Todas las posiciones van referenciadas al origen de coordenadas del entorno. Estas comandas deben ir referidas ademas a la articulación que debe adaptarse a ellas, indicando el robot en \atributo{asset\_name} y la articulación o enlace al que va referenciado en \atributo{body\_name}. Además de esto, se indica el intervalo de tiempo en el que se actualiza esta posición, en este caso 4 segundos, y se indica que el punto sea visible en la simulación. Por último, mediante la variable \atributo{ranges}, se indica el rango de posiciones y rotaciones a generar.

\paragraph{ActionCfg}
Una vez definidos los objetivos del aprendizaje, se definen las acciones a través de su manejador \clase{ActionsCfg}. Este manejador define con cada termino los distintos bloques de acciones; es decir, define el movimiento de las articulaciones. En este caso, (código \ref{lst:accrea}), simplemente se definen los dos tipos de movimientos que tiene cualquier brazo robótico, el movimiento de las articulaciones y el del \emph{gripper}.
\begin{lstlisting}[style=mypython, caption={Definición de los objetivos de entrenamiento mediante la clase \clase{CommandsCfg}},  label={lst:lst:accrea}]
class ActionsCfg:
    """Action specifications for the MDP."""
    arm_action: ActionTerm = MISSING
    gripper_action: ActionTerm | None = None
\end{lstlisting}

\section{Entrenamiento y evaluación}

De una misma manera que en el capítulo anterior, se va realizar el aprendizaje y evaluación de este ejercicio. En primer lugar, se ejecuta un entrenamiento de prueba con pocos entornos para comprobar que estos se generan correctamente. Después de comprobar que esto ocurre (fígura \ref{fig:reachenttst}), se puede proceder al entrenamiento conn un mayor número de entornos, sin necesidad de ejecutar el simulador.

\begin{figure}[ht]
    \label{fig:reachenttst}
    \centering
    \includegraphics[width=\linewidth]{imagenes/reachenttst.png}
    \caption{Prueba para el entrenamiento del robot araña}
\end{figure}

Para este caso se utilizarán 512 entornos. Es interesante notar que al utilizar la forma de manejadores, al descomponer las recompensas en términos individuales, el entrenamiento entrega la información de las recompensas desglosada; a diferencia de la forma directa, donde solo se obtenía el valor de la recompensa global media. Esto nos permite tener más información y poder detectar fácilmente posibles fallos en el entrenamiento. Una vez terminado el entrenamiento (figura \ref{fig:reachenthls}), se observa que la recompensa es negativa. Esto, sin embargo, no es relevante, pues lo importante es que la recompensa se maximize, no que esta sea alta. En este caso, al penalizar en la mayoría de recompensas, lo óptimo es que la recompensa sea próxima a 0, lo cual parece cumplirse. El siguiente paso, será evaluar con \verb|play.py|, de modo que podamos estudiar si cumple los objetivos y si esta recompensa puede minimizarse de alguna forma.

Al evaluar la política de esta manera se observan un problema principal. Por un lado, existen posiciones que requieren que el robot contacte consigo mismo o con el suelo. Esto es un problema, pues puede inhabilitar el movimiento o dañar el robot. Como se espera poder implementar este robot en el robot real, en el próximo apartado se corregirá y mejorará el ejercicio para hacerlo apto para su implementación final.

\section{Mejoras y correcciones}

\chapter{Trabajo dentro del proyecto MetaTool}

Durante este trabajo, la contribución del equipo del proyecto MetaTool ha sido imprescindible. Por un lado, la gran parte del entrenamiento se ha realizado en ordenadores provistos por este equipo, trabajando dentro de su laboratorio en el CAR (Centro de Automática y Robótica, CSIC) \cite{car_csic_upm_car_nodate}. Por otro lado, el investigador principal del laboratorio, Virgilio Gómez Lambo, aporto tanto visión como los distintos objetivos a cumplir dentro del laboratorio. Los objetivos propuestos fueron:
\begin{itemize}
    \item Realización de un ejercicio lifting para una herramienta.
    \item Depuración de un ejercicio para el arrastre con herramienta.
    \item Implementación de un código para el problema de Sim2Sim.
    \item Implementación de un módulo para el problema de Sim2Real.
\end{itemize}

En este primer capítulo, se estudiará en primer lugar el proyecto, tratando de entender su misión y objetivos. Después se estudiará el objetivo de la contribución y la misión propia dentro del proyecto. Por último, se analizarán los dos primeros objetivos, viendo los ejercicios realizados. En estos casos, no se analizará el código completo, sino las partes de el que sean de interés. 

Los dos otros objetivos conformaran el siguiente capítulo, cerrando los contenidos del trabajo para pasar a las conclusiones. Ahora, se va a estudiar la misión y propósito del proyecto global.

\section{Misión y objetivos}
Este proyecto se asienta en una idea central: el uso de la autoevaluación para la creación de herramientas. Se presupone que la creación de herramientas viene derivada de la capacidad humana de comprender su inhabilidad para la realización de ciertas tareas; primero utilizando herramientas naturales (como palos o piedras), para después crear nuevas mejor adaptadas a estas. Esta transición requiere el uso de una serie de herramientas, como por ejemplo la predicción, la meta-cognición, la abstracción y la creatividad; todas ellas asociadas al ser humano. La intención de este proyecto radica en utilizar herramientas de inteligencia artificial para adquirir estas habilidades. Para ello, se tienen tres objetivos principales:
\begin{itemize}
    \item Estudiar las habilidades de meta-cognición y la capacidad de percepción como factores para el desarrollo de la fabricación de herramientas. Cabe especificar que la meta-cognición es la habilidad de auto-evaluar, regular y ser consciente de los procesos cognitivos internos \cite{fleur_metacognition_2021}.
    \item Desarrollar un modelo computacional para la creación de percepción sintética basada en herramientas de meta-cognición y predicción, desarrollando a su vez la fabricación de herramientas.
    \item Validar el modelo anterior mediante herramientas de inteligencia artificial.
\end{itemize}

El proyecto en si abarca una gran cantidad de terreno, sin embargo, el aporte de este trabajo será bastante limitado. En el siguiente apartado, se comentará el enfoque de este trabajo.

\section{Objetivo de la aportación}
Este trabajo de final de grado se centra principalmente en la construcción de entornos y el uso de la herramienta IsaacLab. Por esto, la aportación será limitada a estos conceptos. En vez de trabajar en elementos de percepción, se trabajará sobre el uso de herramientas a través de la inteligencia artificial. 

En primer lugar, se trabajarán sobre los códigos ejemplo de IsaacLab\cite{mittal2025isaaclab} y propios códigos de MetaTool \cite{metatool}, para adaptar los distintos problemas a los objetivos del proyecto. Después, en el siguiente apartado, se diseñará una implementación sim2sim y sim2real para la implementación en robots reales, aplicables al proyecto MetaTool.

Se comenzará en el siguiente apartado con el primer objetivo, el levantamiento de una herramienta.

\section{Levantamiento herramienta}

El primer objetivo de esta parte del trabajo es realizar el levantamiento y agarre de una herramienta. Este ejercicio fue recomendado por los investigadores del proyecto para comenzar a realizar tareas de aprendizaje. Fue uno de los primeros ejercicios realizados. En él, se busca utilizar el robot \emph{Franka}, para levantar un martillo de juguete.

Para la implementación de este ejercicio se utilizaron una serie de recursos. En primer lugar, se utilizó los códigos de la herramienta IsaacLab contenidos en \verb|source/isaaclab_tasks/isaaclab_tasks/manager_based/manipulation/lift|. Estos códigos formulan el problema para un cubo básico. La idea, en este como en gran parte de los ejercicios, es tomar un código para una tarea base; después analizarlo y entenderlo para finalmente modificarlo, adaptando el entorno al objetivo e incluyendo las observaciones y recompensas extras para obtener un resultado exitoso. En segundo lugar, se utilizaron los archivos usd provistos en el \emph{Franka} por IsaacLab, y en el martillo, por MetaTool.

La primera modificación que se realizo fue la sustitución del cubo por la herramienta. Para ello, se alojó el archivo USD en la carpeta de datos del proyecto y se sustituyó el atributo \atributo{usd\_path}, dentro del atributo \atributo{spawn} asociado al elemento de la escena \atributo{object}, para la clase específica \clase{FrankaCubeLiftEnv}, renombrada en el proyecto \clase{FrankaToolLiftEnv}. Una vez sustituido el cubo, se realizó un primer entrenamiento para observar los posibles problemas que produce esta sustitución. Se encontraron principalmente dos: el agarre no se daba en el punto correcto y el robot tocaba el suelo.

Para el primer problema se tomo la ruta más sencilla y eficiente, mover el eje de coordenadas. Para ello, se abrió el archivo de usd del martillo dentro de IsaacSim, seleccionando la malla y desplazando el eje hasta el mango. Al utilizar esta solución se deben tener en cuenta una serie de condicionantes. Pese a que este ejemplo no se integrará en el sim2real, el tomar el mango como punto central de la herramienta sobre el que tomar las recompensas es algo común en otros ejercicios. Esto quiere decir, que al tomar la posición de la herramienta se debe precisar exactamente el punto del cual se quiere agarrar esta. Esto, por otro lado, es un condicionante nuestro, y no derivado del entrenamiento. 

Esto punto fue una de las grandes lecciones derivadas de la colaboración con el proyecto MetaTool. Dentro del aprendizaje por refuerzo hay cierta información que se dan al robot y cierta información que se quiere que aprenda por su cuenta. La ideas que se den al robot limitaran a su vez el rango del aprendizaje. Por ejemplo, al darle el punto de agarre, no prueba distintas posiciones de agarre, las cuales podrían serle beneficiosas en este ejercicio. Sin embargo, si no le damos dicho punto, no se tiene una referencia clara para recompensarle por acercarse a la herramienta; también puede cogerlo desde un punto que no es interesante para el ejercicio. Se debe encontrar el punto exacto donde se maximiza la eficiencia del entrenamiento, intentando limitar su entrenamiento lo mínimo posible.

Para el segundo problema, se decidió crear una nueva recompensa, o en este caso más bien penalización. Para ello, PENDIENTE.

\section{Depuración del arraste con herramienta}

En el segundo problema, se indico desde el proyecto MetaTool que una serie de las recompensas del ejercicio de arrastre no funcionaba correctamente. Se decidió por tanto analizar el ejercicio al completo para poder entenderlo, puesto que se buscaba integrar este ejercicio en el sim2real, y resolver el problema de la recompensa.

El objetivo del entrenamiento es enseñar a un robot a agarrar una herramienta y con ella arrastrar un objeto, en este caso un cubo. Para ello, se utiliza el robot \emph{Robohabilis} \cite{metatool}, mostrado el la figura \ref{fig:robohabilis}. Este robot consiste de una base central a la que se le conectan dos robots \emph{UR3e} \cite{robot_ur3e} y una cámara de visión, la cual no se contemplará en este trabajo.
\begin{figure}[ht]
    \label{fig:robohabilis}
    \centering
    \includegraphics[width=\linewidth]{imagenes/robohabilis.png}
    \caption{Robot RoboHabilis \cite{anton_video_inteligencia_2025}}
\end{figure}

Para esta tarea desde el proyecto MetaTool se diseñaron una serie de recompensas:
\begin{itemize}
    \item \atributo{reaching\_tool}: alcanzar la herramienta con el efector final
    \item \atributo{reaching\_object}: alcanzar la herramienta con la herramienta.
    \item \atributo{lifting\_tool}: levantar la herramienta.
    \item \atributo{grasping\_tool}: agarrar la herramienta.
    \item \atributo{pulling\_object}: arrastrar el objeto.
    \item \atributo{object\_goal\_tracking}: llevar el objeto al objetivo.
    \item \atributo{object\_goal\_tracking\_fine\_grained}: llevar el objeto al objetivo, teniendo en cuenta un mayor grado de precisión. La anterior sería la recompensa por proximidad, esta sería una recompensa extra por alcanzarlo.
    \item \atributo{action\_rate}: penalización por el uso de acciones.
    \item \atributo{joint\_vel}: penalización por la cantidad de movimiento.
\end{itemize}
Las funciones que determinan el cálculo de estas recompensas, \atributo{func}, vienen definidas en el archivo \verb|source/MT_ext/MT_ext/tasks/manipulation/pull_object/mdp/rewards.py|

De todas estas recompensas, donde se encontró un problema fue en la recompensa para el arrastre de la herramienta, \atributo{object\_is\_pulled}. Se observa dentro del entrenamiento, que a pesar de no estar realizándose el arrastre, la recompensa se da en todo momento, obteniendo un promedio de alrededor de 9'7 sobre 10. Esto es un problema, ya que no permite evaluar el arrastre correctamente.

Esta recompensa se obtiene mediante la función \metodo{object\_is\_pulled(...)}. Esta función se muestra a continuación:
\begin{lstlisting}[style=mypython, caption={Definición de la función para el cálculo de la recompensa por arrastre.},  label={lst:objpullrewdef}]
def object_is_pulled(
    env: ManagerBasedRLEnv,
    std: float, 
    tool_cfg: SceneEntityCfg = SceneEntityCfg("tool"),
    object_cfg: SceneEntityCfg = SceneEntityCfg("object"),
    object_contact_sensor_cfg: SceneEntityCfg = SceneEntityCfg("object_contact_sensor"),
    tool_contact_sensor_cfg: SceneEntityCfg = SceneEntityCfg("tool_contact_sensor"),
) -> torch.Tensor:
    """Reward the agent for grasping the object."""
    tool: RigidObject = env.scene[tool_cfg.name]
    object: RigidObject = env.scene[object_cfg.name]
    object_contact_sensor: ContactSensor = env.scene[object_contact_sensor_cfg.name]
    tool_contact_sensor: ContactSensor = env.scene[tool_contact_sensor_cfg.name]
    # Check robot tool active contact (current_contact_time > 0)
    tool_contact_active = (tool_contact_sensor.data.current_contact_time.squeeze(-1) > 0).float()
    # Check tool object active contact (current_contact_time > 0)
    object_contact_active = (object_contact_sensor.data.current_contact_time.squeeze(-1) > 0).float()
    # Get positions of tool and object
    tool_pos_w = tool.data.root_pos_w # Target object position: (num_envs, 3)
    object_pos_w = object.data.root_pos_w # End-effector position: (num_envs, 3)
    object_tool_distance = torch.norm(tool_pos_w - object_pos_w, dim=1) # Distance of the end-effector to the object: (num_envs,)
    # Determine if the object is near the tool
    # object_tool_distance = torch.where(object_tool_distance > minimal_distance, object_tool_distance, 0.0)
    valid_contact = (tool_contact_active * object_contact_active).bool()
    object_tool_distance = torch.where(valid_contact, 0.0, object_tool_distance)
    # Compute distance-based reward component
    distance_reward = 1 - torch.tanh(object_tool_distance / std)
    #Combine contact presence and proximity reward
    total_reward = object_contact_active * distance_reward * tool_contact_active
    # total_reward = object_contact_active * tool_contact_active
    return total_reward
\end{lstlisting}
Antes de evaluar el problema, se va a analizar como se calcula la recompensa. Este análisis será también importante para ver un ejemplo de como se calculan las recompensas normalizadas.

En primer lugar, se extraen los distintos elementos del entorno. Entre ellos se encuentran la herramienta, el objeto y sus respectivos sensores. Seguidamente se comprueba el contacto en ambos sensores, utilizando el atributo \atributo{current\_contac\_time} de la clase \clase{ContactSensor} \cite{isaaclab_api}. El sensor de la herramienta se activa cuando hay un contacto entre la herramienta y el robot, mientras que para el objeto se activa con un contacto entre la herramienta y el objeto. Después se procede a calcular, la distancia entre el objeto y la herramienta. Añadido a este calculo, se supone que cuando hay un contacto activo, es decir, el robot toca la herramienta y la herramienta el objeto, la distancia es 0. Esta distancia después se normaliza con \metodo{torch.tanh()} \cite{pytorch_docs}, obteniendo valores entre -1 y 1; y en el caso de 0, 1 como recompensa de la distancia. Esta recompensa se multiplica después por los contactos activos y se entrega finalmente.

A primera se puede ver que el cálculo de la posición no afecta en sí al cálculo de la recompensa, pues para obtener recompensa debe haber contacto activo y siempre que haya contacto activo se obtendrá una distancia de 0 y recompensa de 1. Por tanto, este cálculo se puede omitir. Sin embargo, esta no es la raíz del problema.

Habiendo realizado una depuración en tiempo de compilación, se observó que los sensores de contacto tienen un ruido. Este ruido hace que el sensor se active a pesar de no encontrar un contacto real. Para resolver este problema se introdujo una nueva variable dentro de los sensores de contacto, \atributo{force\_threshold}. Esta modificación se realiza dentro de la clase específica de configuración, \clase{RobohabilisCubePullEnvCfg}, definida dentro del archivo \verb|source/MT_ext/MT_ext/tasks/manipulation/pull_object/config/robohabilis/joint_pos_env_cfg.py|. De este modo, se limita la fuerza con la cual el sensor se activa, evitando que salte con el ruido. Sin embargo, al volver a depurar el código, se observó como seguía activándose. Esto era debido a que la variable \atributo{contact\_alive} no tiene en cuenta este límite. Por esto, se utilizo la variable \atributo{net\_forces\_w}, la cual si se ve afectada por esta variable. De este modo se solucionó al completo el problema, resultando en el siguiente código:

PENDIENTE DE INCLUIR.

CONCLUSIONES PENDIENTES.


\chapter{El problema Sim2Real}
El problema Sim2Real hace referencia a la diferencia entre el rendimiento de una política en el simulador y en el mundo real. Estas diferencias pueden deberse a multiples factores. Puede presentarse por la mala representación de características del entorno; también por la percepción del robot, que puede hacer variar las observaciones \cite{da_survey_2025}. Estos y otros factores hacen que sea un problema común para todos los ejercicios de aprendizaje por refuerzo.

En este capítulo se presentarán algunas soluciones para el problema. Después, se seleccionará aquella que se adapte mejor al trabajo, implementando la solución dentro del ejercicio de entrenamiento. A continuación, se implementará una nueva simulación dentro de IsaacSim. En esta nueva simulación, no se usarán las herramientas de IsaacLab. El objetivo es probar la política en un entorno de simulación externo al probado. El siguiente paso será la implementación en el robot real. Para ello, se ha diseñado un modulo en python usando herramientas de UR-RTDE \cite{noauthor_universalrobotsrtde_python_client_library_2025}. Por último, se realizará un ensayo, implementando una política en un robot.

\section{Enfoques}
El primer punto que se debe estudiar del problema Sim2Real son los distintos enfoques con los que se puede trabajar para minimizar su efecto. La manera más eficaz de poder adaptar a la implementación real es aleatorizar ciertos parámetros que puedan diferir en el mundo real. De este modo, se puede entrenar la política para afrontarse a variaciones en dichos parámetros. Teniendo en cuenta esto, surgen distintos tipos de aprendizaje \cite{sim2realpaper}:
\begin{itemize}
    \item Simulación ideal: no se incluye ningún parámetro aleatorizado.
    \item \emph{Fine-tunning}: primero se entrena en una simulación ideal. Después, se identifican los parámetros que necesitan ser aleatorizados y se entrena para cada uno de ellos por separado.
    \item Curriculum: de nuevo, entrena primero en una simulación ideal e identifica los parámetros a aleatorizar. En este caso, los parámetros se van incluyendo uno a uno, manteniendo los anteriores; finalizando por tanto con un entrenamiento en el que se tienen en cuenta todos los parámetros aleatorizados.
    \item Ideal a aleatorio: primero se entrena con una simulación ideal y después se vuelve a entrenar con todos los parámetros aleatorizados.
    \item Aleatorización del dominio: se entrena desde una simulación que contempla desde el principio todas las variables aleatorizadas.
\end{itemize}

En \emph{"Analysis of Randomization Effects on Sim2Real Transfer in
Reinforcement Learning for Robotic Manipulation Tasks"} \cite{sim2realpaper} se realizó un estudio de estos distintos tipos de entrenamiento. De este mismo articulo se obtuvo los enfoques anteriores. El que obtuvo el mejor resultado fue el aprendizaje con aleatorización  del dominio, seguido por \emph{fine-tunning}. Debido a ser el mejor modo, y tener tiempo limitado dentro del entrenamiento, se escoge esta opción a seguir.

En el próximo apartado se estudiará como incluir esta herramienta en el aprendizaje, concretamente en el ejercicio de empuje (pues en el de alcance viene integrado en el ejemplo).

\section{Aleatorización del dominio.}
La aleatorización del dominio consiste en proveer a la simulación de una serie de variaciones en el entrenamiento para generalizar la política en el mundo real \cite{tobin_domain_2017}. Para poner un ejemplo de esta implementación se introducirán para el ejemplo de empuje una variabilidad en algunas de sus observaciones y eventos.

Dentro de las observaciones, la clase \clase{ObsTerm} permite introducir ruido a la observación mediante el parámetro del constructor \atributo{noise}. Esta ruido funcionará como la variabilidad del sistema. Para este, se utilizará la clase \clase{GaussianNoiseCfg}. Esta clase recibe la siguiente serie de parámetros en su constructor \cite{isaaclab_api}:
\begin{itemize}
    \item \atributo{mean}: la media del ruido. Esto nos permite incluir un desfase al parámetro. En este caso, la media será 0, pues no contamos con ninguna desviación en los medidores.
    \item \atributo{std}: desviación estándar del ruido.
    \item \atributo{operation}: operación para incluir el ruido
\end{itemize}
Este tipo de ruido se incluirá a las observaciones de la velocidad, la posición del objeto y la posición de la herramienta.

Por otro lado, dentro del manejador de eventos tenemos distintas formas de introducir variabilidad al sistema. Esto se realiza mediante el parámetro \atributo{func}. En este ejemplo se han incluido dos funciones capaces de variar aspectos relevantes:
\begin{itemize}
    \item \atributo{randomize\_joint\_parameters}: varía la posición de las articulaciones desde las cuales se empieza.
    \item \atributo{randomize\_rigid\_body\_material}: varía los parámetros de los objetos físicos.
    \item \atributo{randomize\_rigid\_body\_mass}: varía la masa de los objetos rígidos en escena.
\end{itemize}

Con esta variabilidad incluida se ha analizado como incluir aleatorización del dominio y preparado el sistema para una posible implementación. A continuación se estudiará como realizar una simulación externa a la herramienta IsaacLab.

\section{Implementación en IsaacSim}
A continuación se va estudiar la implementación del sistema en IsaacSim. Esta implementación es de desarrollo propio, empleando la clase \clase{PolicyController} y \clase{ConfigLoader} entre otras herramientas de IsaacSim \cite{NVIDIA_IsaacSim}. El código se encuentra dentro del proyecto general, incluido en los anexos. A su vez, el código viene inspirado en los ejemplos de IsaacSim, implementando una estructura distinta ya que estos vienen en una extensión \cite[isaacsim.examples.interactive]{NVIDIA_IsaacSim}.

El primer archivo relevante a esta implementación se encuentra en \verb|source/ARMetaToolPG/ARMetaToolPG/assets/policys/policy_controllers/robohabilis.py|. Este archivo declara una clase \clase{RobohabilisPullObjectPolicy} que hereda de la clase \clase{PolicyController}. Esta clase encarga de cargar y manejar la política. A continuación, se van a estudiar el objetivo de sus funciones, que serán a su vez sus responsabilidades:
\begin{itemize}
    \item \metodo{\_\_init\_\_(...)}: se encarga de de inicializar los principales atributos de la clase, como los objetos de la escena (entregados por el constructor) y el robot (creado y definido en la clase base); así como cargar la política, almacenada en este caso en \verb|source/ARMetaToolPG/ARMetaToolPG/assets/policys/policy_pull_object_rh|. Este método se define en la clase específica
    \item \metodo{load\_policy(...)}: se encarga de cargar la política. Es usado en el constructor para este fin. Este método queda definido en la clase base.
    \item \metodo{\_compute\_observations(...)}: se encarga de calcular las observaciones, almacenándolas en el parámetro \atributo{obs}. Este método se define en la clase específica.
    \item \metodo{\_compute\_action(...)}: se encarga de calcular las acciones utilizando la política. Este método queda definido dentro de la clase base.
    \item \metodo{forward(...)}: se encarga de aplicar las acciones al robot. Para ello se debe extraer la información de las acciones y preparar la indicación de la posición. Se debe recordar que en este caso se trabaja con dos tipos de acciones, binarias y de posición.
    \item \metodo{initialize(...)}: se encarga de inicializar el robot y los objetos en la escena.
\end{itemize}

Esta clase después se almacena como atributo de otra clase, \clase{RoboHabilisTask}, la cual se define para agrupar los procesos de la tarea a ejecutar en IsaacSim. Esta clase se define en \verb|scripts/sim2sim/load_robohabilis.py|, el archivo que se ejecutará dentro de IsaacSim mediante $Window \rightarrow ScripEditor$. Esta clase tiene las siguientes funciones:
\begin{itemize}
    \item \metodo{\_\_init\_\_(...)}: se encarga de limpiar el mundo existente y crear uno nuevo. Caber resaltar que IsaacSim sigue la misma estructura que IsaacLab en los elementos de la simulación; estructura vista en el apartado \ref{ap:structisaac}.
    \item \metodo{set\_up\_scene(...)}: se encarga de definir los elementos de la simulación. El robot se define dentro de \clase{RobohabilisPullObjectPolicy}, mientras que los objetos mediante la clase \clase{RigidObject} \cite[isaacsim.core.experimental.prims]{NVIDIA_IsaacSim}.
    \item \metodo{load\_world\_async(...)}: se encarga de cargar el mundo e inicializar las físicas de simulación. Por último, llama a la función \metodo{setup\_post\_load(...)}, que veremos a continuación.
    \item \metodo{setup\_post\_load(...)}: se encarga de cargar la llamada recurrente al método \metodo{on\_physics\_step(...)}, así como inicializar los distintos elementos. Por último, llama al método del mundo \atributo{play\_async(...)}, que mantiene el bucle a la llamada recurrente.
    \item \metodo{on\_physics\_steps}: se encarga de llamar a la función \metodo{forward()}, la cual avanza la simulación.
\end{itemize}

Dentro de este archivo también se puede encontrar la función que define el bucle asíncrono. Para ello, se usa la biblioteca \api{asyncio} de python \cite{python_docs}. La llamada a esta función permite ejecutar el programa dentro de IsaacSim sin bloquear sus procesos internos. Para la ejecución del código, se utiliza la función \metodo{load\_robohabilis()}. Este método instancia la clase anterior, crea la escena con el método \metodo{se\_up\_scene()} y termina llamando a la función \metodo{load\_world\_async()}.

Con esto, se puede ejecutar una simulación externa al aprendizaje. En este trabajo, este script ha sido de gran utilidad. No solo para la evaluación de la política, dónde es un paso clave en su implementación, sino también para explorar posiciones de robot, utilizando la clase \clase{SingleArticulation} dentro del controlador. Por otro lado, la evaluación de la posible implementación de la política de empuje, pues permite tener un fácil acceso a las posiciones de la herramienta y el objeto. En el siguiente apartado, se estudiará la implementación en el robot real, utilizando un cliente python \cite{noauthor_universalrobotsrtde_python_client_library_2025} y un robot UR3 como servidor, controlado desde el cliente python.

\section{Implementación en robot real}

Para la implementación en el robot real se van a utilizar dos lenguajes de programación. En primer lugar, utilizando python y la herramientas de Universal Robots de RTDE (Real-Time Data Exchange). Esta parte de la aplicación se encargará de cargar la política, recibir el estado del robot, calcular la acción siguiente y enviar dicha acción al robot. El robot se encarga de gestionar los movimientos del robot, regulando y sincronizando el proceso, preparar el robot para la ejecución del movimiento y aplicar las acciones que lo conforman.

En este apartado, se estudiará cada código por encima. En el cliente python se analizará el código en su conjunto mediante el diagrama de clases y se explicará como se maneja la comunicación con el robot real.  Para el programa del robot se analizará el código al completo.

\subsection{Cliente Python}

El cliente python es el encargado de gestionar la política y enviar la información de las acciones al robot. Este cliente se conforma a través de un módulo de python desarrollado en este trabajo, bajo el nombre \api{sim2real}. Este módulo contiene varios ejercicios de implementación. En este apartado se estudiará uno de ellos, enfocado al ejercicio \emph{Reach}. Para ello, se analizará el diagrama de clases creado y mostrado en la figura \ref{fig:UMLsim2real}.

\begin{figure}[h!]
    \centering
    \includegraphics[width=0.8\textwidth]{imagenes/Sim2Real.pdf}
    \caption{Diagrama de clases del módulo sim2real aplicado al ejercicio \emph{Reach}.}
    \label{fig:UMLsim2real}
\end{figure}

La pieza central de este diagrama es la clase \clase{EnviromentAdapter}. Esta clase se encarga de almacenar y gestionar el robot, la política y sus interacciones. Esta clase no es instanciable, sirve simplemente como base para los casos específicos. Esta clase generaliza parte de la construcción de las clases específicas, almacenando la política y la interfaz del robot; así como la gestión de las interacciones entre ambas, definidas en el método \metodo{step}. De esta clase hereda el adaptador específico para el caso del \emph{Reach}, la clase \emph{ReachUR3}. 

La política queda almacena en el atributo \atributo{controlador}, a través de una clase \atributo{PolicyController}. Esta clase se ha extraído de un ejemplo de referencia recomendado por el proyecto MetaTool, un ejemplo para el alcance de un robot \emph{Kinova} \cite{Le_Lay_Kinova_Gen3_RL_2025}. A su vez, este es un código adaptado de la clase \clase{PolicyController} antes utilizada \cite{NVIDIA_IsaacSim}, extrayendo la creación del robot. En este módulo, la clase se vuelve a modificar, extrayendo el cálculo de observaciones que se realizará en la clase \clase{EnviromentAdapter}. El objetivo de esto es que esta clase funcione directamente como una red neuronal, a la que entregamos observaciones y devuelve acciones.

El robot, por otro lado, queda almacenado en el atributo \atributo{robot}, a través de la clase \clase{IRobot}. Esta clase define una serie de métodos para el control de este. Por un lado, su constructor inherente, que se encarga de crear la conexión. Por otro lado, el método \metodo{get\_state(self)}, encargado de recibir la información y devolver una clase de datos con esta. Otro método que se debe definir es \metodo{send\_action(self)}, encargado de gestionar el envío de la acción al robot. Por último, se debe definir el método \metodo{\_disconnect}, encargado de gestionar la desconexión del robot. De esta clase, hereda la clase específica para el UR3, la cual define cada uno de estos métodos, utilizando herramientas de UR-RTDE \cite{noauthor_universalrobotsrtde_python_client_library_2025}. Estas herramientas permiten realizar una conexión con el robot e intercambiar información. La información intercambiada viene especificada en un fichero .xml. El cliente python recibe del robot la posición de las articulaciones (\atributo{actual\_q}), la velocidad de las articulaciones (\atributo{actual\_qd}) y un registro de salida (\atributo{output\_int\_register\_0}). Por otro lado, el robot recibe información dentro de sus registros de entrada, la acción a realizar mediante las posiciones de las articulaciones objetivo y un \atributo{watchdog}, que junto con el registro de salida permiten sincronizar los dos procesos. 

De este modo, este módulo cliente de python permite intercambiar información con el robot y gestionar las interacciones de este con la política. A continuación, se estudiará como se utiliza esta información para crear un bucle de control en el robot, utilizando URscript.

\subsection{Servidor robot.}

\section{Ensayos}
\subsection{Prueba simulada}
\subsection{Prueba real}


\chapter{Conclusiones Finales}

El trabajo realizado presenta el marco teórico del aprendizaje por refuerzo y su implementación mediante la herramienta IsaacLab, todo ello enfocado a la robótica. El aprendizaje por refuerzo es una de las tecnologías punteras en el control de robots. Este trabajo es una base para la implementación de esta disciplina dentro de la escuela, siendo relevante para futuros trabajos. En este capítulo se analizará el aporte de este trabajo y los resultados obtenidos.

\section{Resultados}

Al comienzo del trabajo se presentaron una serie de objetivos: dos objetivos principales y ocho secundarios. En esta sección se analizará el cumplimiento de estos objetivos, comenzando por los objetivos secundarios, para después presentar los principales.

\begin{itemize}
\item Objetivo 1: Se realizó un estudio del arte. En este, se presentan la relación del aprendizaje por refuerzo con la robótica y distintos artículos que muestran aplicaciones prácticas en distintas disciplinas.
\item Objetivo 2: Se realizó una extensa base teórica del aprendizaje por refuerzo. En ella, se presentó la definición y la estructura del aprendizaje, su base matemática junto con los MDPs y los distintos algoritmos aplicables. Estos se presentan de una forma didáctica, con el objetivo de usar la documentación como material académico.
\item Objetivo 3: Se realizó un análisis profundo de la herramienta IsaacLab. Se presentó la base de la herramienta, abarcando su estructura, las distintas formas de programación, las estructuras de datos y los principales procesos de entrenamiento y evaluación.
\item Objetivo 4: Se realizaron dos análisis sobre ejercicios prácticos, abarcando ambas formas de programación e implementando mejoras en ambos.
\item Objetivo 5: Se trabajó dentro del proyecto MetaTool y se presento su misión y objetivos.
\item Objetivo 6: Se presentaron algunos de los ejercicios prácticos trabajados y las correcciones implementadas.
\item Objetivo 7: Se estudió el problema Sim2Real y se presentaron distintas soluciones, implementando finalmente la aleatorización del dominio.
\item Objetivo 8: Se diseño un código para la implementación de políticas, obteniendo resultados aptos en simulación y limitados en la realidad.
\end{itemize}

Teniendo en cuenta estos objetivos secundarios, se presenta una evaluación de los objetivos principales. Por un lado, la creación de una base teórica a quedado cubierta en este trabajo, mediante los objetivos 1, 2, 3 y 4. El resultado en su conjunto es bueno, habiendo construido una base teórica sólida para futuros trabajos. Por otro lado, la implicación dentro del proyecto MetaTool ha sido limitada, debido a su alta complejidad. Sin embargo, tomando los objetivos 5, 6, 7 y 8, se ha dado cierta involucración que, aunque no haya tenido un efecto en el proyecto general, ha sido beneficiosa para los conocimientos y contenidos aportados. Por tanto, se han cumplido los objetivos con éxitos, asumiendo ciertas limitaciones en la segunda parte del trabajo.

\chapter{Aportaciones del trabajo}

Añadido a los objetivos de este trabajo, este trabajo contiene más aportaciones dignas de resaltar y que complementan estos objetivos. En esta sección se presentarán, dividiendo de nuevo el trabajo en dos partes marcadas: teoría (objetivos 1 a 4) y práctica (objetivos 5 a 8).

La parte teórica, por un lado, no es únicamente una presentación de conceptos. El capítulo 3 sigue una progresión propia de este trabajo. Esta permite seguir los conceptos progresivamente y manteniendo una cohesión. Si bien está estructura está influenciada por los principales materiales utilizados \cite{silver_lectures_nodate} \cite{sutton_reinforcement_2020}, no responde a ninguna de estos dos, sino se vuelve una combinación de ambos y de la propia experiencia de aprendizaje. Además, la mayor parte de este material se encuentra en lengua inglesa, por lo que la traducción al español se vuelve valiosa. El capítulo 4, por otro lado, realiza una labor parecida, agrupando toda la información contenida dentro de la documentación oficial \cite{isaaclab_doc}. La información contenida en este capítulo se reparte en distintos apartados. En este capítulo se agrupa toda esta información, teniendo en cuenta una aplicación didáctica, al igual que en el apartado 3. Por último, los análisis profundos de cada ejercicio práctico son completamente propios de este trabajo, derivados de multiples horas de análisis e investigación dentro de las APIs de la herramienta \cite{isaaclab_api}. Estos análisis permiten facilitar la compresión del conjunto global de entrenamiento, que supone la gran parte del trabajo realizado y permite acortar el tiempo dedicado a esto en futuros proyectos.

La parte práctica, por otro lado, se encuentra más limitada, pero no carece en absoluto de aportaciones. La principal de estas son los distintos módulos implementados para la implementación de políticas. Los módulos existentes actualmente son extremadamente limitados a aplicaciones concretas. El módulo implementado permite una gran flexibilidad a la hora de diseñar entornos, adaptándose a la modularidad de IsaacLab. A su vez, los resultados obtenidos en la práctica, a pesar de ser deficientes, son una de las grandes aportaciones en este trabajo. En la mayoría de artículos expuestos se indican únicamente casos de éxito. Siendo el primer trabajo referente a esta disciplina dentro de la universidad estos resultados sirven de referencia para futuras aplicaciones.

\section{Limitaciones}

Pese a las aportaciones expuestas, se han encontrado ciertas limitaciones. Por un lado, en la parte teórica, la principal limitación de este trabajo es en el construcción de algoritmos propios. Esta limitación no es derivada del trabajo sino del enfoque. La construcción de algoritmos parte de la base teórica expuesta, pero requiere de una mayor investigación acerca de redes neuronales y algoritmos modernos de aprendizaje profundo. En este trabajo se han expuesto las bases de estos, pero realizar un estudio extenso requiere de sentido gracias a las bibliotecas de algoritmos contenidas dentro IsaacLab. La limitación en la parte práctica, sin embargo, es mayor. Los problemas encontrados en el control del robot limitaron la implementación del ejercicio \emph{Reach}, así como la implementación de ejercicios más complicados dentro de MetaTool, implementados sobre el \emph{Robohabilis}.

Las limitaciones del trabajo no suponen un fracaso para la creación de la base teórica realizada. Esta base permite planear futuros proyectos, que expandan sobre esta base.

\section{Trabajos futuros}

La base expuesta y las limitaciones encontradas abren camino para futuros trabajos que expandan sobre el desarrollo de este. En esta sección se proponen posibles enfoques para nuevos trabajos:
\begin{itemize}
    \item \textbf{Implementación de políticas en UR3}. La disponibilidad de estos robots en el laboratorio de la ETSIDI permiten realizar gran número de pruebas en ellos. Sin embargo, los distintos problemas encontrados en el control de estos mediante políticas ha sido el principal limitador de este trabajo. Se propone un trabajo en el que se utilicen directamente las políticas generadas en este u otro trabajo pare enfocarse en su implementación; y no en la generación como en este caso.
    \item \textbf{Generación de políticas con otras herramientas}. IsaacLab no es la única herramienta de aprendizaje automático, existen otras aplicaciones y herramientas que disponen de estas funcionalidades. IsaacLab es la más potente, pero su uso de las tarjetas gráficas es muy limitante. La base teórica del aprendizaje por refuerzo, el módulo de implementación de políticas y las ejemplificaciones en IsaacLab son extrapolables a cualquier caso de aprendizaje.
    \item \textbf{Aplicación de políticas en robots}. Este trabajo, combinado a los anteriores expuestos, permitirían tener la base suficiente para llevar a cabo proyectos enfocados únicamente a la aplicación de las políticas en robos reales. Este tipo de proyectos debería dividirse en dos trabajos separados. Por un lado, la generación de los elementos a simular y sus políticas; y por el otro, la implementación Sim2Real en el robot real.
    \item \textbf{Diseño de nuevos algoritmos}. Este tipo de proyectos divergen del terreno de la escuela ETSIDI, estando más relacionados con programación que con implementaciones ingenieriles. No obstante, la base teórica presentada supone un gran punto de inicio para cualquier trabajo de estas características.
\end{itemize}

El objetivo de próximos trabajos debería ser la consolidación de resultados reales, algo en lo que en este trabajo no se ha podido alcanzar. Para ello se realiza una recomendación: utilizar este trabajo como base teórica y enfocar los esfuerzos en un único problema. Este trabajo, debido a su amplitud teórica, se enfoca en múltiples problemas en un solo trabajo. Esto requiere gran cantidad de tiempo, pues cada ejemplo práctico o ejercicio práctico tienen una serie de problemas individuales. Esta multitud de problemas individuales consumen gran cantidad de horas, por lo que es prioritario consolidar los resultados una vez superados.

\bibliographystyle{unsrt}
\bibliography{bibliografia/citas}
\appendix
\clearpage
\section*{Anexo A. Repositorios de código}
\addcontentsline{toc}{section}{Anexo A. Repositorios de código}

El código desarrollado en el presente trabajo se encuentra disponible en los siguientes repositorios de GitHub:

\begin{itemize}
  \item \textbf{Repositorio 1 - Proyecto General de IsaacLab}: Proyecto general con todos los códigos del trabajo relevantes a la herramienta IsaacLab y la aplicación IsaacSim.\\
  \url{https://github.com/EdeAntonio/ARMetaToolPG}

  \item \textbf{Repositorio 2 - Proyecto Sim2Real}:Proyecto con el módulo para la implementación de las políticas generadas.\\
  \url{https://github.com/EdeAntonio/Sim2RealEdAS}
\end{itemize}

Estos repositorios contienen el código empleado para el entrenamiento de la red neuronal, así como las herramientas utilizadas para la evaluación experimental.

\section*{Anexo B. Vídeos de resultados experimentales}
\addcontentsline{toc}{section}{Anexo B. Vídeos de resultados experimentales}

Debido a las limitaciones del formato escrito, los resultados experimentales se complementan con una serie de vídeos que muestran el comportamiento del sistema durante los ensayos de alcance.

Los vídeos se encuentran disponibles en la siguiente carpeta de Google Drive:

\begin{center}
    \url{https://drive.google.com/drive/folders/19nxncHIfXZ1wsK8j33GQqheYxs2wF4jY?usp=drive_link}
\end{center}

En dichos vídeos se pueden observar tanto los casos representativos como situaciones límite del método propuesto.

\section*{Anexo C. Resultados completos de los ensayos}
\addcontentsline{toc}{section}{Anexo C. Resultados completos de los ensayos}

La figura \ref{fig:tabla_ensayos} recoge el conjunto completo de resultados obtenidos en los ensayos de alcance realizados. 
Con el objetivo de preservar la legibilidad y el formato visual, la tabla se incluye como imagen en orientación horizontal.

\begin{landscape}
\begin{figure}[!ht]
    \label{fig:tabla_ensayos}
    \centering
    \includegraphics[width=\linewidth]{imagenes/tabla_errores.pdf}
    \caption{Resultados completos de los ensayos de alcance.}
\end{figure}
\end{landscape}

\end{document}

