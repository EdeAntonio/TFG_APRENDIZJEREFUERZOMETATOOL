\chapter{Introducción}

\section{Contexto y motivación}
Según la página oficial de \emph{Nvidia} \cite{Nvd_def_RL}, empresa considerada una de las mayores impulsoras de esta disciplina, el aprendizaje por refuerzo es una técnica de aprendizaje automático que permite a los robots tomar decisiones basadas en la experiencia. En este trabajo de final de grado se estudia esta doctrina; con el objetivo de entender sus conceptos fundamentales y desarrollar (mediante la herramienta de \emph{Nvidia}, \emph{IsaacLab}) distintos entornos capaces de implementar este procedimiento.

La inteligencia artificial es actualmente una tecnología puntera con una gran cantidad de aplicaciones. Concretamente, el aprendizaje automático se ha aplicado en disciplinas como la medicina, en la generación de reportes de imágenes médicas; las finanzas, en la gestión de órdenes de compra; o la energía, en sistemas de refrigeración de servidores \cite{RL_aplications}. En la robótica concretamente, ha tomado un gran protagonismo. En esta disciplina, grandes empresas del sector, como \emph{Boston Dynamics} han empezado a utilizar aprendizaje automático en múltiples tareas \cite{BD_RLusage}.

El interés en este proyecto surge de la posibilidad de aplicar esta herramienta dentro del proyecto \emph{ROMERIN}, un robot modular escalador destinado a la inspección de infraestructuras \cite{Romerin_Descrip}.  Debido a la complejidad del aprendizaje por refuerzo, se identificó la necesidad de realizar un estudio completo. Esto, añadido a la cesión de recursos del proyecto \emph{MetaTool} en la formación, llevó a la colaboración dentro de este proyecto, centrada en el análisis y la revisión de código. 

Antes de comenzar el trabajo, en este capítulo, se presentarán los objetivos y contenidos del mismo. De esta forma, se obtendrá una visión clara de las ideas principales y la estructura del documento. En el siguiente apartado, se enumerarán los principales objetivos de este trabajo.

\section{Objetivos del trabajo}
Este proyecto tiene como objetivo realizar un estudio del aprendizaje por refuerzo y la herramienta IsaacLab, para lo cual se fijan dos objetivos principales. En primer lugar, asentar una base teórica sólida tanto del aprendizaje por refuerzo como de la herramienta IsaacLab. Se espera que futuros estudiantes puedan basarse en ella para realizar trabajos en esta disciplina. En segundo lugar, realizar labores dentro del proyecto europeo MetaTool, con el fin de adquirir experiencia práctica.

Para alcanzar estos objetivos, se proponen una serie de objetivos específicos:
\begin{enumerate}
    \item Obtener una visión general del impacto del aprendizaje por refuerzo en el campo de la robótica.
    \item Estudiar las bases del aprendizaje por refuerzo, centrándose en su estructura, fundamentos matemáticos y sus principales algoritmos.
    \item Explicar el funcionamiento de la herramienta \emph{IsaacLab} para su aplicación en el ámbito del aprendizaje por refuerzo.
    \item Analizar ejemplos de dicha herramienta, para así profundizar en ella y proporcionar una guía práctica para trabajos futuros.
    \item Realizar un estudio del proyecto \emph{MetaTool}: misión y visión.
    \item Realizar trabajos prácticos en el proyecto con la herramienta \emph{IsaacLab}
    \item Estudiar el problema \emph{Sim2Real} y sus posibles soluciones
    \item Desarrollar un código para la implementación de políticas.
\end{enumerate}

Para el cumplimiento de estos objetivos, se deberá tener en cuenta todo lo que se abarcará y cómo dicho alcance se adapta a los objetivos propuestos. En el siguiente apartado, e desarrollará este análisis.. 

\section{Alcance y limitaciones}
En este TFG se cubre el proceso para realizar entornos de aprendizaje por refuerzo. Al tener este objetivo en mente, en este TFG se abordarán distintos aspectos de esta disciplina. Este trabajo contempla desde los fundamentos del aprendizaje, hasta las distintas estructuras de datos, clases y ficheros empleados para la ejecución y simulación de entornos.

En primer lugar, para contextualizar el trabajo dentro del aprendizaje por refuerzo en robótica, se revisará el estado actual del arte. Se presentarán los avances más importantes en distintos campos de la robótica, entre ellos la manipulación, la locomoción y otras aplicaciones como drones, navegación o dispositivos de visión.

A continuación, se explicará la teoría fundamental del aprendizaje por refuerzo. En primer lugar, se presentará la estructura principal que se utiliza en esta disciplina y sus partes, entre las que se encuentran los agentes, los entornos y sus interacciones (acciones, observaciones y recompensas). Dentro de este marco teórico se presentarán los procesos de decisión de Markov (MDP), en los cuales se fundamentan los algoritmos que se utilizarán. Una vez estudiados estos conceptos, se presentarán diversos algoritmos, desde los más simples (\emph{Monte Carlo}, \emph{TD}) hasta aquellos que se utilizarán en las simulaciones (\emph{PPO}, \emph{SAC}, \emph{A2C}).

Una vez desarrollado el marco teórico, se procederá a introducir la herramienta \emph{IsaacLab}. Después de una primera introducción a la herramienta, se comenzará a explicar sus distintas funcionalidades. Primero, se explicará qué es la herramienta y cómo se estructuran las simulaciones dentro de la aplicación. Después, se analizarán las dos principales arquitecturas de los entornos: la forma directa y la forma basada en manejadores. Definidas las arquitecturas, se estudiarán las principales estructuras de datos: las clases y los tensores. Por último, se estudiará cómo, una vez definidos los entornos, se realiza el aprendizaje y se evalúa el resultado final.

A continuación, se analizarán dos casos prácticos de la herramienta \emph{IsaacLab}, cada uno construido a partir de una arquitectura diferente. Para ambos ejemplos, primero se presentará la tarea escogida y la motivación que justifica esta elección. Seguidamente, se presentará el diagrama de clases que describe el entorno a entrenar. Este diagrama de clases se diseccionará, analizando cada una de ellas, con especial interés en sus atributos y métodos. Una vez analizado el entorno, se estudiará la ejecución del aprendizaje, valorando después el resultado final del proceso. Por último, se presentarán algunas mejoras posibles dentro del ejercicio.

Habiendo estudiado las distintas características del aprendizaje por refuerzo (RL) y la herramienta, se desarrollará la participación dentro del proyecto MetaTool. En este capítulo, se definirá el contexto del proyecto \emph{MetaTool} y cuál es su principal objetivo. Seguidamente, se concretarán las tareas en las cuales se intervendrá y el objetivo de la participación. Para cada caso abordado, se expondrán los problemas afrontados y las soluciones tomadas.

Otro punto importante del aprendizaje por refuerzo que se estudiará será el problema del Sim2Real, que consiste en la aplicación de las políticas entrenadas en simulación para el control en el entorno real. Primero, se presentará el concepto de Sim2Real y sus principales desafíos. Seguidamente, se enumerarán y analizarán las distintas técnicas para realizar este trasvase a la realidad. Finalmente, se desarrollara código para la implementación de las políticas en robots reales.

En la última parte del trabajo, se expondrán las conclusiones del proyecto en su conjunto, valorando las aportaciones realizadas, las dificultades encontradas y las oportunidades para trabajos futuros.

Este trabajo, debido a la gran extensión de esta disciplina, se dejan fuera algunos paradigmas. Por un lado, únicamente se estudia, dentro del aprendizaje automático, el aprendizaje por refuerzo, dejando fuera el aprendizaje supervisado y no supervisado. Además, aunque se emplee aprendizaje por refuerzo profundo, no se profundizará en la base matemática de sus algoritmos, así como en las bibliotecas que los implementan. Se estudiarán las características de las redes neuronales, pero no se profundizará en su formulación matemática.

Este será el alcance completo del trabajo. Sin embargo, se debe tener en cuenta un punto más antes de comenzar con las tareas prácticas: la metodología. En el siguiente apartado se cubrirá este tema.

\section{Metodología}
En este trabajo de final de grado se distinguen principalmente dos líneas: una parte teórica acerca del aprendizaje por refuerzo y una parte práctica mediante programación en Python y URscript. Por otro lado, la preparación de este documento se ha realizado después de ocho meses de trabajo en tareas de programación e investigación de forma independiente o en conjunto con el equipo de investigación del proyecto MetaTool. A continuación, se expondrá la metodología característica de cada apartado.

La introducción, en primer lugar, se ha preparado tras la realización de la mayoría de las labores teóricas y prácticas del trabajo. De este modo, al disponer de una visión general del conjunto del trabajo, se han expuesto las distintas características del mismo y su enfoque general.

Para el estado del arte, con el objetivo de mostrar una visión general del estado del aprendizaje por refuerzo en la robótica, se presentan distintos artículos de investigación sobre esta disciplina y sus aplicaciones en la práctica. Al realizar este ejercicio después de un periodo de aprendizaje y práctica, se han podido determinar los factores más importantes de cada artículo, así como seleccionar los artículos más relevantes.

En el apartado tres, en el cual se exponen los fundamentos teóricos del aprendizaje por refuerzo, se siguió la siguiente metodología. En primer lugar, se estudió un curso de aprendizaje por refuerzo impartido por David Silver \cite{silver_lectures_nodate}. Mediante este curso se obtuvo una visión general de esta disciplina, entendiendo su estructura general y la fundamentos de sus algoritmos. Una vez obtenida una visión general, y tras aplicar estos conocimientos en labores prácticas, se estudió de forma más específica cada elemento, consultando distintas fuentes de información. Cabe resaltar, dentro de estas fuentes, el libro sobre aprendizaje por refuerzo de Sutton y Barto \cite{sutton_reinforcement_2020}. Sobre este se fundamenta la gran mayoría de definiciones formales. 

El cuarto apartado, acerca de la herramienta \emph{IsaacLab}, se abordó de manera distinta. Al ser \emph{IsaacLab} una herramienta concreta y propiedad de una entidad privada, NVIDIA, existe menos diversidad de información acerca de ella. Por tanto, para su estudio, se utilizó principalmente la información contenida en sus fuentes oficiales. En este apartado concreto, se utilizaron principalmente los tutoriales proporcionados por la plataforma para el aprendizaje de su estructura y aplicación, así como los distintos glosarios de las funciones y clases, así como la información detallada acerca de la propia plataforma y sus bibliotecas. A esto se le sumó el conocimiento adquirido en el trabajo en conjunto con el equipo MetaTool, en especial con Virgilio Gómez, especialista de la plataforma y líder de la división en la que se trabajó.

Los apartado quinto y sexto, al constar de un análisis concreto de un ejemplo proporcionado por la plataforma, se utilizaron los conocimientos adquiridos en el apartado anterior. Para realizar este análisis se ha seguido el siguiente proceso. En primer lugar, se analizó todo el ejercicio en su conjunto, estudiando los distintos ficheros que intervienen en la ejecución. Con esta estructura identificada, se realizó un diagrama de clases, detallando los distintos atributos y las funciones utilizadas. Con este diagrama en mente, se estudió cada clase por sí sola, explicando la función de cada apartado del código y su aportación global al ejercicio.

A su vez, en el apartado séptimo se debe tratar de manera distinta, pues se trata de tareas dentro de un proyecto externo. Por ello, antes de analizar el ejercicio se realizó un pequeño estudio del proyecto general. Con el enfoque general en mente, se llevó a cabo un estudio de las tareas realizadas. A diferencia del apartado anterior, este estudio se basó en la depuración realizada del código, en el cual se llevaron a cabo una serie de correcciones para su correcto funcionamiento.

En el apartado octavo, enfocado a la implementación en el robot de políticas, se estudió un caso real de esta implementación \cite{Le_Lay_Kinova_Gen3_RL_2025}, aunque aplicado a un robot distinto. Estudiado este caso, se diseñó un programa para su uso en el \emph{UR3} \emph{Robohabilis} \cite{metatool} \cite{robot_ur3e}, utilizando el lenguaje \emph{URScript} \cite{URScriptLanguage} para el control del robot y algunas clases derivadas del caso estudiado.

Por último, una vez terminado el trabajo, se expondrán las conclusiones obtenidas del mismo, basadas en la experiencia obtenida en el transcurso del proyecto.

En resumen, este trabajo sigue distintas metodologías, dependiendo de si se trata de un enfoque práctico o teórico. Considerando el conjunto del trabajo, se podría definir una metodología general. Primero, se realiza el estudio teórico, tanto del aprendizaje por refuerzo en general como de la herramienta específica. Después, se utilizan los conocimientos obtenidos para realizar estudios prácticos y analíticos.

Toda la metodología y alcance descritos se concretan en este documento. A continuación, se presenta cómo se encuentra estructurado.

\section{Estructura del documento}

El documento se estructura en capítulos y una serie de anexos. Los capítulos se agrupan en dos partes diferenciadas. En primer lugar, del capítulo 2 al 4, se introducen las bases teóricas relevantes para la ejecución de ejercicios de entrenamiento: el estado del arte, el marco teórico del aprendizaje por refuerzo y un estudio en profundidad de la herramienta IsaacLab. En segundo lugar, los capítulos 5, 6, 7 y 8 recogen las prácticas realizadas en el proyecto MetaTool y la implementación de una política. Estos capítulos vienen precedidos de una introducción, ya desarrollada, y cerrados por un capítulo de conclusiones finales. Aparte de la documentación, se incluyen una serie de anexos con los materiales relevantes para el trabajo: repositorios, material audiovisual y datos de ensayos.

Presentados el objetivo, la metodología, el alcance y la estructura del documento, se continúa con el desarrollo del trabajo, dando comienzo con el estado del arte.