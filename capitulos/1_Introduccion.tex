\chapter{Introducción}

\section{Contexto y motivación}
Según la página oficial de \emph{Nvidia} \cite{Nvd_def_RL}, una de las mayores impulsoras de esta disciplina, el aprendizaje por refuerzo es una técnica de aprendizaje automático que permite a los robots tomar decisiones basadas en la experiencia. En este trabajo de final de grado se va a estudiar esta doctrina; con el objetivo de entender sus conceptos fundamentales y poder crear (mediante la herramienta de \emph{Nvidia}, \emph{IsaacLab}) distintos entornos capaces de ejecutar este procedimiento.

Las inteligencias artificiales son actualmente una tecnología puntera con una gran cantidad de aplicaciones. Concretamente, el aprendizaje automático se ha aplicado en disciplinas como la medicina, en la generación de reportes de imágenes médicas; como las finanzas, en la reserva de órdenes de compra; o como la energía, en sistemas de refregamiento de bancos de datos \cite{RL_aplications}. En la robótica concretamente, ha tomado un gran protagonismo. En esta disciplina, grandes empresas del sector como \emph{Boston Dynamics} han empezado a utilizar aprendizaje automático en múltiples tareas \cite{BD_RLusage}.

El interés en este proyecto nace de la idea de aplicar esta herramienta dentro del proyecto \emph{ROMERIN}, un robot modular escalador para la inspección de infraestructuras \cite{Romerin_Descrip}.  Debido a la complejidad del aprendizaje por refuerzo, se vio la necesidad de realizar un estudio completo. Esto, añadido a la cesión de recursos del proyecto \emph{MetaTool} en la formación, llevo a colaborar dentro de este proyecto, analizando y revisando código. 

Antes de comenzar el trabajo, en este capítulo, se estudiarán los objetivos y contenidos de este trabajo. De esta forma, se obtendrá una visión clara de las ideas principales y la estructura del documento. En el siguiente apartado, se enumeraran los principales objetivos de este trabajo.

\section{Objetivos del trabajo}
Este proyecto busca realizar un estudio del aprendizaje por refuerzo y la herramienta IsaacLab, para lo cual se fijan dos objetivos principales. En primer lugar, asentar una base teórica fuerte tanto del aprendizaje por refuerzo como la herramienta IsaacLab. Se pretende que futuros estudiantes puedan basarse en ella para realizar trabajos en esta disciplina. En segundo lugar,  realizar labores dentro del proyecto europeo MetaTool; utilizando estas para ganar experiencia.

Para alcanzar estos objetivos, se proponen una serie de objetivos secundarios:
\begin{enumerate}
    \item Obtener una visión general del impacto del aprendizaje por refuerzo en el campo de la robótica.
    \item Estudiar las bases del aprendizaje por refuerzo, centrándose en su estructura, base matemática y sus principales algoritmos.
    \item Explicar el funcionamiento de la herramienta \emph{IsaacLab} para su aplicación en aprendizaje por refuerzo.
    \item Analizar ejemplos de dicha herramienta, para  así profundizar en ella y proveer de una guía práctica para trabajos futuros.
    \item Estudio del proyecto \emph{MetaTool}: Misión y Visión.
    \item Realización de trabajos prácticos en el proyecto con la herramienta \emph{IsaacLab}
    \item Estudio del problema \emph{Sim2Real} y posibles soluciones
    \item Realización de un código para la implementación de políticas.
\end{enumerate}

Para el cumplimiento de estos objetivos, se deberá tener en cuenta todo lo se va abarcar y cómo este alcance se adapta a los objetivos propuestos. En el siguiente apartado, se realizará esto mismo. 

\section{Alcance y limitaciones}
En este TFG cubriremos el proceso para realizar entornos de aprendizaje automático. Al tener este objetivo en mente, en este TFG se podrán encontrar distintos aspectos de esta disciplina. Este trabajo contempla desde los aspectos más fundamentales de la teoría del aprendizaje automático, hasta las distintas estructuras de datos, clases y ficheros que ejecutan y simulan los entornos.

Primero de todo, para situarse dentro del marco del Aprendizaje por Refuerzo en robótica, se revisará el estado actual del arte. Se presentarán los avances más importantes en distintos campos de la robótica; entre ellos la manipulación, la locomoción y otras aplicaciones como drones, navegación o dispositivos de visión.

A continuación, se explicará la teoría fundamental del Aprendizaje por Refuerzo. En una primera instancia, se presentará la estructura principal que se utiliza en esta disciplina y sus partes, entre las que se encuentran los agentes, los entornos y sus interacciones (acciones, observaciones y recompensas). Dentro de este marco teórico se estudiará los procesos de decisión Markov (MDP), en los cuales se asienta la base de los algoritmos que se utilizarán. Una vez estudiado esto, se presentarán algunos de estos algoritmos, desde los más simples (\emph{Monte Carlo}, \emph{TD}) hasta los que se utilizarán en las simulaciones (\emph{PPO}, \emph{SAC}, \emph{A2C}).

Una vez desarrollado el marco teórico, se procederá a introducir la herramienta \emph{IsaacLab}. Después de una primera introducción a la herramienta, se comenzará a explicar sus distintas funcionalidades. Primero, se explicará qué es y cómo se estructuran las simulaciones dentro de la aplicación. Después, se analizarán las dos principales arquitecturas de los entornos: la forma directa y la basada en manejadores. Definidas las arquitecturas, se estudiarán las principales estructuras de datos: las clases y los tensores. Por último, se estudiará como, una vez definidos los entornos, se realiza el aprendizaje y cómo se evalúa el resultado final.

A continuación, se analizará dos casos prácticos de la herramienta \emph{IsaacLab}; cada uno construido a partir de una arquitectura diferente. Para ambos ejemplos, primero, se presentará la tarea escogida y la motivación detrás de esta elección. Seguidamente, se presentará el diagrama de clases que describe el entorno a entrenar. Este diagrama de clases se diseccionará, analizando cada una de las clases, con especial interés en sus atributos y métodos. Desmenuzado el entorno, se estudiará la ejecución del aprendizaje, valorando después el resultado final de esta. Por último, se presentarán algunas mejoras posibles dentro del ejercicio.

Habiendo estudiado las distintas características del RL y la herramienta, se comenzará a intervenir dentro del proyecto MetaTool. En este capítulo, se definirá el contexto del proyecto \emph{MetaTool} y cuál es su principal objetivo. Seguidamente, se concretará las tareas en las cuales se intervendrá y el objetivo de la participación. Para cada caso realizado, se expondrán los problemas afrontados y las soluciones tomadas.

Otro punto importante del aprendizaje por refuerzo que se estudiará será el problema del Sim2Real, que consiste en la aplicación de las políticas entrenadas para el control en el entorno real. Primero, se presentará el concepto de Sim2Real y sus principales desafíos. Seguidamente, se enumerarán y analizarán las distintas técnicas para realizar este trasvase a la realidad. Finalmente, se preparará un código para la implementación de las políticas en robots reales.

En la última parte del trabajo, se expondrán las conclusiones del proyecto en su conjunto, valorando las aportaciones realizadas, las dificultades encontradas y las oportunidades de trabajos futuros.

Este trabajo, debido a la gran extensión de esta disciplina, se dejan de cubrir algunos paradigmas. Por un lado, únicamente se estudia dentro del aprendizaje automático el aprendizaje por refuerzo, dejando fuera el aprendizaje supervisado y no supervisado. Además, pese a que se realice aprendizaje por refuerzo profundo, no se entrará en detalle en la base matemática de sus algoritmos, así como las bibliotecas implementadas con estos. Se estudiarán las características de las redes neuronales, pero no se profundizará en la matemática detrás de ellas.

Este será el alcance completo del trabajo. Sim embargo, se debe tener en cuenta un punto más antes de comenzar con las tareas prácticas: la metodología. En el siguiente apartado se cubrirá este tema.

\section{Metodología}
En este trabajo de final de grado existen principalmente dos líneas: una parte teórica acerca del aprendizaje por refuerzo y una parte práctica mediante programación en Python y finalmente URscript. Por otro lado, la preparación de este documento se ha realizado después de 8 meses realizando tareas de programación e investigación por propia cuenta; o en conjunto con el equipo de investigación del proyecto MetaTool. A continuación, se expondrá la metodología característica de cada apartado.

La introducción, en primer lugar, se ha preparado después de haber realizado la mayoría de las labores teóricas y prácticas del trabajo. Teniendo así una visión general del trabajo global, se han expuesto las distintas características de este y su enfoque general.

Para el estado del arte, al querer mostrar una visión general del estado del aprendizaje por refuerzo en la robótica, se han buscado distintos artículos de investigación sobre esta disciplina y sus aplicaciones prácticas. Al haber realizado este ejercicio después de este periodo de aprendizaje y práctica, se han podido identificar los factores más importantes de cada artículo, así como identificar los artículos más relevantes.

En cuanto al apartado 3, en el cual se exponen los fundamentos teóricos del aprendizaje por refuerzo, se ha seguido la siguiente metodología. En primer lugar, se estudió un curso de aprendizaje por Refuerzo impartido por David Silver \cite{silver_lectures_nodate}. Mediante este curso se obtuvo una visión general de esta disciplina, entendiendo su estructura general y la base para sus algoritmos. Una vez obtenida una visión general, y después de aplicar esta visión en labores prácticas, se estudió más específicamente cada elemento, indagando en distintas fuentes de información. Cabe resaltar dentro de estas fuentes, el libro sobre aprendizaje por refuerzo de Sutton y Barto \cite{sutton_reinforcement_2020}. Sobre este se trabajan la gran mayoría de definiciones formales. 

El 4º apartado, acerca de la herramienta \emph{IsaacLab}, se trató de manera distinta. Al ser \emph{IsaacLab} una herramienta concreta y propiedad de una entidad privada, Nvidia, existe menos diversidad de información acerca de ella. Por tanto, para su estudio, se utilizó principalmente la información contenida en sus fuentes oficiales. En este apartado concreto, se utilizaron principalmente los tutoriales proporcionados por la plataforma para el aprendizaje de su estructura y aplicación, así como los distintos glosarios de las funciones y clases; y su información detallada acerca de la propia plataforma y sus bibliotecas. A esto se le sumo el conocimiento aprendido en el trabajo en conjunto con el equipo MetaTool, en especial con Virgilio Gómez, especialista de la plataforma y líder de la división en la que se trabajó.

El apartado 5 y 6, al constar de un análisis concreto de un ejemplo proporcionado por la plataforma, se han utilizado los conocimientos aprendidos en el apartado anterior. Para realizar este análisis se ha seguido el siguiente proceso. En primer lugar, se analizó todo el ejercicio en su conjunto, estudiando los distintos ficheros que se ponen en ejecución. Con esta estructura identificada, se realizó un diagrama de clases, identificando los distintos atributos y las funciones utilizadas. Con este diagrama en mente, se estudió cada clase por si sola, explicando la función de cada apartado del código y su aportación global al ejercicio.

A su vez el apartado 7, se debe tratar de manera distinta, pues se trata de tareas dentro de un proyecto externo. Por ello, antes de analizarse el ejercicio se realizó un pequeño estudio del proyecto general. Con el enfoque general en mente, se realizó otro estudio de las tareas realizadas. A diferencia del apartado anterior, este estudio se basará en la depuración realizada del código, en el cual se realizaron una serie de correcciones para su correcto funcionamiento.

El apartado 8, enfocado a la implementación en el robot de políticas, se estudió un caso real de esta implementación \cite{Le_Lay_Kinova_Gen3_RL_2025}, pero en un robot distinto. Estudiado este caso, se diseño un programa para su uso en el \emph{UR3} \emph{Robohabilis} \cite{metatool} \cite{robot_ur3e}, utilizando el lenguaje \emph{URScript} \cite{URScriptLanguage} para el control del Robot y algunas clases implementadas en el caso estudiado.

Por último, una vez terminado el trabajo, se expondrá las conclusiones obtenidas de este, basándose plenamente en la experiencia obtenida en el transcurso del proyecto.

En conclusión, este trabajo sigue distintas metodologías, dependiendo si se trata de un enfoque práctico o teórico. Tomando en conjunto todo, se podría definir una metodología general. Primero, se realiza el estudio teórico, tanto del aprendizaje en refuerzo general como el de la herramienta específica. Después, se utiliza los conocimientos obtenidos para realizar estudios prácticos y analíticos.

Toda la metodología y alcance descrito se concretan en este documento. A continuación, se explicará como se encuentra estructurado.

\section{Estructura del documento}
PENDIENTE