\section*{Anexo A. Repositorios de código}
\addcontentsline{toc}{section}{Anexo A. Repositorios de código}

El código desarrollado en el presente trabajo se encuentra disponible en los siguientes repositorios de GitHub:

\begin{itemize}
  \item \textbf{Repositorio 1 - Proyecto General de IsaacLab}: Proyecto general con todos los códigos del trabajo relevantes a la herramienta IsaacLab y la aplicación IsaacSim.\\
  \url{https://github.com/EdeAntonio/ARMetaToolPG}

  \item \textbf{Repositorio 2 - Proyecto Sim2Real}:Proyecto con el módulo para la implementación de las políticas generadas.\\
  \url{https://github.com/EdeAntonio/Sim2RealEdAS}
\end{itemize}

Estos repositorios contienen el código empleado para el entrenamiento de la red neuronal, así como las herramientas utilizadas para la evaluación experimental.

\section*{Anexo B. Vídeos de resultados experimentales}
\addcontentsline{toc}{section}{Anexo B. Vídeos de resultados experimentales}

Debido a las limitaciones del formato escrito, los resultados experimentales se complementan con una serie de vídeos que muestran el comportamiento del sistema durante los ensayos de alcance.

Los vídeos se encuentran disponibles en la siguiente carpeta de Google Drive:

\begin{center}
    \url{https://drive.google.com/drive/folders/19nxncHIfXZ1wsK8j33GQqheYxs2wF4jY?usp=drive_link}
\end{center}

En dichos vídeos se pueden observar tanto los casos representativos como situaciones límite del método propuesto.

\section*{Anexo C. Resultados completos de los ensayos}
\addcontentsline{toc}{section}{Anexo C. Resultados completos de los ensayos}

La figura \ref{fig:tabla_ensayos} recoge el conjunto completo de resultados obtenidos en los ensayos de alcance realizados. 
Con el objetivo de preservar la legibilidad y el formato visual, la tabla se incluye como imagen en orientación horizontal.

\begin{landscape}
\begin{figure}[!ht]
    \label{fig:tabla_ensayos}
    \centering
    \includegraphics[width=\linewidth]{imagenes/tabla_errores.pdf}
    \caption{Resultados completos de los ensayos de alcance.}
\end{figure}
\end{landscape}